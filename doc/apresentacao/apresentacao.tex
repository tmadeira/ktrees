\documentclass{beamer}

\usepackage[utf8]{inputenc}
\usepackage[brazil]{babel}
\usepackage{amsmath,amssymb,amsfonts,amsthm}
\usepackage{graphics}
\usepackage{hyperref}
\usepackage{tikz}

\usetikzlibrary{positioning,shapes,arrows}

\usetheme{Copenhagen}

\title[Geração unif. de \emph{$k$-trees} para aprendizado de redes bayesianas]{\LARGE Geração uniforme de \emph{$k$-trees} para aprendizado de redes bayesianas}
\author[Tiago Madeira {\tt <madeira@ime.usp.br>}]{
  {\Large Tiago Madeira}\\
  {\footnotesize \tt <madeira@ime.usp.br>}}
\institute[IME-USP]{{\normalsize Supervisor: Prof. Dr. Denis Deratani Mauá}\\
  \ \\
  Bacharelado em Ciência da Computação\\
  Instituto de Matemática e Estatística\\
  Universidade de São Paulo}
\date{Novembro de 2016}

\begin{document}
  \frame{\titlepage}

  \section{Introdução}

  \subsection{No que consiste o trabalho?}

  \begin{frame}
    \frametitle{No que consiste o trabalho?}

    Estudo sobre amostragem uniforme de \emph{$k$-trees} e seu uso no aprendizado da estrutura de redes bayesianas com \emph{treewidth} limitado.
  \end{frame}

  \subsection{Por que estudar $k$-trees?}

  \begin{frame}
    \frametitle{Por que estudar \emph{$k$-trees}?}

    Há interesse considerável em desenvolver ferramentas eficientes para manipular \emph{$k$-trees}, porque \textbf{problemas NP-difíceis são resolvidos em tempo polinomial} em \emph{$k$-trees} e subgrafos de \emph{$k$-trees}.

    \vspace{1em}

    Alguns exemplos\footnote{\scriptsize Stefan Arnborg, Andrzej Proskurowski. Linear time algorithms for NP-Hard problems restricted to partial $k$-trees. \emph{Discrete Applied Mathematics}, 23:11--24, 1989.}:

    \begin{itemize}
      \item Encontrar tamanho máximo dos conjuntos independentes;
      \item Computar tamanho mínimo dos conjuntos dominantes;
      \item Calcular número cromático;
      \item Determinar se tem um ciclo hamiltoniano.
    \end{itemize}
  \end{frame}

  \subsection{Por que gerar $k$-trees?}

  \begin{frame}
    \frametitle{Por que gerar \emph{$k$-trees?}}

    Há muitas razões, como por exemplo para testar a eficácia de algoritmos aproximados.

    \vspace{1em}

    O problema que desperta nosso interesse é o \textbf{aprendizado de redes bayesianas}.
  \end{frame}

  \subsection{O que foi feito?}

  \begin{frame}
    \frametitle{O que foi feito?}

    \begin{itemize}
      \item Implementação do algoritmo de Caminiti \emph{et al.} (2010)\footnote{\scriptsize Severio Caminiti, Emanuele G. Fusco, Rossella Petreschi. Bijective linear time coding and decoding for $k$-trees. \emph{Theory of Computing Systems}, 46:284--300, 2010.} para \textbf{codificar \emph{$k$-trees} de forma bijetiva em tempo linear}.
      \item Implementação de algoritmo para \textbf{amostrar \emph{$k$-trees} uniformemente} e testes para comprovar seu funcionamento.
      \item Estudo sobre \textbf{aprendizado de redes bayesianas com \emph{treewidth} limitado} por meio da amostragem uniforme de \emph{$k$-trees} conforme artigo de Nie \emph{et al.} (2014)\footnote{\scriptsize Siqi Nie, Denis D. Mauá, Cassio P. de Campos, Qiang Ji. Advances in learning bayesian networks of bounded treewidth. \emph{CoRR}, abs/1406.1411, 2014.}.
      \item \textbf{Comparação entre métodos} para aprender redes bayesianas.
    \end{itemize}
  \end{frame}

  \subsection{Onde encontrar o trabalho?}

  \begin{frame}
    \frametitle{Onde encontrar o trabalho?}

    Código (desenvolvido em \emph{Go}\footnote{\scriptsize \url{https://golang.org/}}) e documentação: \url{https://github.com/tmadeira/tcc/}
  \end{frame}

  \section{Geração uniforme de $k$-trees}

  \subsection{O que são $k$-trees?}

  \begin{frame}
    \frametitle{Primeiramente, o que são \emph{$k$-trees}?}

    Uma \textbf{\emph{$k$-tree}} é definida da seguinte forma recursiva\footnote{\scriptsize Frank Harary, Edgar M. Palmer. On acyclic simplicial complexes. \emph{Mathematika}, 15:115--122, 1968.}:

    \begin{itemize}
      \item Um grafo completo com $k$ vértices é uma \emph{$k$-tree}.
      \item Se $T_k' = (V, E)$ é uma \emph{$k$-tree}, $K \subseteq V$ é um $k$-clique e $v \not \in V$, então $T_k = (V \cup \{v\}, E \cup \{(v,x) \ | \  x \in K\})$ é uma \emph{$k$-tree}.
    \end{itemize}

    \begin{figure}
      \begin{minipage}{0.3333\textwidth}
        \centering
        \scalebox{0.5}{
          \begin{tikzpicture}
              [scale=.6,auto=left,every node/.style={draw, circle, inner sep = 0pt, minimum width = 0.72cm}]
            \node (n1) at (3,8) {1};
            \node (n2) at (1.5,6) {2};
            \node (n3) at (4.5,6) {3};
            \node (n4) at (2.25,4) {4};

            \foreach \from/\to in {n1/n2, n1/n3, n2/n4}
              \draw (\from) edge (\to);
          \end{tikzpicture}
        }

        (a)
      \end{minipage}\begin{minipage}{0.3333\textwidth}
        \centering
        \scalebox{0.5}{
          \begin{tikzpicture}
              [scale=.6,auto=left,every node/.style={draw, circle, inner sep = 0pt, minimum width = 0.72cm}]
            \node (n1) at (3,8) {1};
            \node (n2) at (5,8) {2};
            \node (n3) at (2.5,6) {3};
            \node (n4) at (6,6) {4};
            \node (n5) at (3.5,4) {5};

            \foreach \from/\to in {n1/n2, n3/n1, n3/n2, n4/n1, n4/n2, n5/n3, n5/n2}
              \draw (\from) edge (\to);
          \end{tikzpicture}
        }

        (b)
      \end{minipage}\begin{minipage}{0.3333\textwidth}
        \centering
        \scalebox{0.5}{
          \begin{tikzpicture}
              [scale=.6,auto=left,every node/.style={draw, circle, inner sep = 0pt, minimum width = 0.72cm}]
            \node (n1) at (3,8) {1};
            \node (n2) at (5,8) {2};
            \node (n3) at (7,8) {3};
            \node (n4) at (4,6) {4};
            \node (n5) at (5,4) {5};

            \foreach \from/\to in {n1/n2, n2/n3, n1/n4, n2/n4, n3/n4, n4/n5, n3/n5}
              \draw (\from) edge (\to);

            \draw (n1) edge [bend left=40] (n3);
            \draw (n1) edge [bend right=40] (n5);
          \end{tikzpicture}
        }

        (c)
      \end{minipage}

      \caption{
        \textbf{(a)} Uma \emph{$1$-tree} (ou seja, uma árvore comum) com $4$ vértices.
        \textbf{(b)} Uma \emph{$2$-tree} com $5$ vértices.
        \textbf{(c)} Uma \emph{$3$-tree} com $5$ vértices.
      }
      \label{fig:ktree}
    \end{figure}
  \end{frame}

  \begin{frame}
    \frametitle{\emph{$k$-trees} enraizadas}

    Uma \textbf{\emph{$k$-tree} enraizada} é uma \emph{$k$-tree} com um $k$-clique destacado $R = \{r_1, r_2, \cdots, r_k\}$ que é chamado de \textbf{raiz} da \emph{$k$-tree} enraizada.

    \begin{figure}
      \begin{minipage}{0.5\textwidth}
        \centering
        \scalebox{0.5}{
          \begin{tikzpicture}
              [scale=.6,auto=left,every node/.style={draw, circle, inner sep = 0pt, minimum width = 0.72cm}]
            \node (n10) at (1,9) {10};
            \node (n2) at (2.5,7) {2};
            \node (n1) at (1,4) {1};
            \node (n5) at (3,2.75) {5};
            \node (n7) at (2,1) {7};
            \node (n9) at (4.5,9.5) {9};
            \node (n6) at (4,5) {6};
            \node (n4) at (6,10.5) {4};
            \node (n3) at (8.5,6.5) {3};
            \node (n8) at (8,4.5) {8};
            \node (n11) at (9,9) {11};

            \foreach \from/\to in {n1/n2, n1/n5, n1/n7, n1/n8, n2/n3, n2/n5, n2/n6, n2/n8, n2/n9, n2/n10, n2/n11, n3/n4, n3/n5, n3/n8, n3/n9, n3/n10, n3/n11, n4/n9, n4/n11, n5/n7, n5/n8, n6/n8, n6/n9, n7/n8, n8/n9, n9/n10, n9/n11}
              \draw (\from) edge (\to);
          \end{tikzpicture}
        }

        (a)
      \end{minipage}\begin{minipage}{0.5\textwidth}
        \centering
        \scalebox{0.5}{
          \begin{tikzpicture}
              [scale=.6,auto=left,every node/.style={draw, circle, inner sep = 0pt, minimum width = 0.72cm}]
            \node (n10) at (1,8.25) {10};
            \node[fill=gray!30] (n2) at (1.5,10.5) {2};
            \node (n1) at (2,3.75) {1};
            \node (n5) at (3,6) {5};
            \node (n7) at (3.5,1.5) {7};
            \node[fill=gray!30] (n9) at (4.5,10.5) {9};
            \node (n6) at (6,6) {6};
            \node (n4) at (9,4) {4};
            \node[fill=gray!30] (n3) at (7.5,10.5) {3};
            \node (n8) at (4.5,8.25) {8};
            \node (n11) at (8,8.25) {11};

            \foreach \from/\to in {n1/n5, n1/n7, n2/n5, n2/n8, n2/n9, n2/n10, n2/n11, n3/n8, n3/n9, n3/n10, n3/n11, n4/n9, n4/n11, n5/n7, n5/n8, n6/n8, n6/n9, n8/n9, n9/n10, n9/n11}
              \draw (\from) edge (\to);

            \draw (n1) edge [bend right=20] (n8);
            \draw (n1) edge [bend left=50] (n2);
            \draw (n2) edge [bend left] (n3);
            \draw (n2) edge [bend right=20] (n6);
            \draw (n3) edge [bend left] (n4);
            \draw (n3) edge [bend left=20] (n5);
            \draw (n7) edge [bend right=20] (n8);
          \end{tikzpicture}
        }

        (b)
      \end{minipage}

      \caption{
        \textbf{(a)} Uma \emph{$3$-tree} $T_3$ com 11 vértices.
        \textbf{(b)} A mesma \emph{$3$-tree} ($T_3$) enraizada no $3$-clique $\{2, 3, 9\}$.
      }
      \label{fig:rootedktree}
    \end{figure}
  \end{frame}

  % TODO: utilidade, treewidth

  \subsection{Codificação de $k$-trees}

  \begin{frame}
    \frametitle{A relação entre geração e codificação}

    O problema de gerar \emph{$k$-trees} está intimamente relacionado ao problema de codificá-las e decodificá-las. De fato, se há uma codificação bijetiva que associa \emph{$k$-trees} a \emph{strings}, basta gerar \emph{strings} uniformemente aleatórias para gerar \emph{$k$-trees} uniformemente aleatórias.
  \end{frame}

  \begin{frame}
    \frametitle{Codificação de \emph{$k$-trees}}

    \begin{itemize}
      \item Em 1889, Cayley\footnote{\scriptsize Arthur Cayley. A theorem on trees. \emph{Quart J. Math}, 23:376--378, 1889.} demonstrou que para um conjunto de $n$ vértices existem $n^{n-2}$ árvores possíveis. Desde lá, foram criados vários códigos para árvores, como o de Prüfer\footnote{\scriptsize Heinz Prüfer. Neuer beweis eines satzes über permutationen. \emph{Archiv der Mat. und Physik}, 27:142--144, 1918.}.
      \item Em 1970, Rényi e Renýi apresentaram uma codificação redundante (ou seja, não bijetiva) para um subconjunto de \emph{$k$-trees} rotuladas que chamamos de \emph{$k$-trees} de Rényi\footnote{\scriptsize C. Rényi, A. Rényi. The prüfer code for $k$-trees. \emph{Combinatorial Theory and its Applications}, 945--971, 1970.}. Definição: Uma \textbf{\emph{$k$-tree} de Rényi} $R_k$ é uma \emph{$k$-tree} enraizada com $n$ vértices rotulados em $[1, n]$ e raiz $\{n-k+1, \cdots, n\}$.
    \end{itemize}
  \end{frame}

  \begin{frame}
    \frametitle{A solução de Caminiti \emph{et al.}}

    \begin{itemize}
      \item Apenas em 2008 surgiu um código bijetivo para \emph{$k$-trees} com algoritmos lineares de codificação e decodificação. Esses algoritmos, propostos por Caminiti \emph{et al.}, foram implementados neste trabalho.
      \item O código é formado por uma permutação de tamanho $k$ e uma generalização do \emph{Dandelion Code}\footnote{\scriptsize Ömer Eğecioğlu, J. B. Remmel. Bijections for cayley trees, spanning trees, and their q-analogues. \emph{Journal of Combinatorial Theory}, 42:15--30, 1986.}. A codificação das \emph{$k$-trees} associa elementos em $\mathcal{T}^n_k$ (conjunto das \emph{$k$-trees} com $n$ vértices) com elementos em:

        $$
        \mathcal{A}^n_k = { [1,n] \choose k } \times (\{ ( 0, \varepsilon ) \} \cup ([1,n-k] \times [1,k]))^{n-k-2}
        $$
    \end{itemize}
  \end{frame}

  \begin{frame}
    \frametitle{A solução de Caminiti \emph{et al.}}

    % TODO: mostrar transformações
  \end{frame}

  \subsection{Geração uniforme}

  \begin{frame}
    \frametitle{Geração uniforme de \emph{$k$-trees}}

    % TODO: algoritmo, implementação, Go
  \end{frame}

  \subsection{Testes}

  \begin{frame}
    \frametitle{Testes}

    % TODO: testes
  \end{frame}

  \section{Aprendizado de redes bayesianas}

  \subsection{Redes bayesianas}

  \begin{frame}
    \frametitle{Redes bayesianas}

    % TODO: definição, exemplos
  \end{frame}

  \subsection{Motivação}

  \begin{frame}
    \frametitle{Aprendizado de redes bayesianas}

    % TODO: Aprendizado de redes bayesianas: descrição
  \end{frame}

  \subsection{Aprendizado por amostragem de $k$-trees}

  \begin{frame}
    \frametitle{Aprendizado por amostragem de \emph{$k$-trees}}

    % TODO: descrição do algoritmo
  \end{frame}

  \subsection{Experimentos}

  \begin{frame}
    \frametitle{Experimentos}

    % TODO: resultados
  \end{frame}

  \section{Considerações finais}

  \subsection{Conclusões}

  \begin{frame}
    \frametitle{Conclusão}

    % TODO: escrever
  \end{frame}

  \subsection{Agradecimentos}

  \begin{frame}
    \frametitle{Agradecimentos}

    % TODO: escrever
  \end{frame}

  \begin{frame}
    \frametitle{Perguntas?}

    % TODO: escrever
  \end{frame}
\end{document}
