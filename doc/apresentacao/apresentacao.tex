\documentclass{beamer}

\usepackage[utf8]{inputenc}
\usepackage[brazil]{babel}
\usepackage{graphics}
\usepackage{hyperref}

\usetheme{Copenhagen}

\title[Geração unif. de \emph{$k$-trees} para aprendizado de redes bayesianas]{\LARGE Geração uniforme de \emph{$k$-trees} para aprendizado de redes bayesianas}
\author[Tiago Madeira {\tt <madeira@ime.usp.br>}]{
  {\Large Tiago Madeira}\\
  {\footnotesize \tt <madeira@ime.usp.br>}}
\institute[IME-USP]{{\normalsize Supervisor: Prof. Dr. Denis Deratani Mauá}\\
  \ \\
  Bacharelado em Ciência da Computação\\
  Instituto de Matemática e Estatística\\
  Universidade de São Paulo}
\date{Novembro de 2016}

\begin{document}
  \frame{\titlepage}

  \section{Introdução}

  \subsection{No que consiste o trabalho?}

  \begin{frame}
    \frametitle{No que consiste o trabalho?}

    Estudo sobre amostragem uniforme de \emph{$k$-trees} e seu uso no aprendizado da estrutura de redes bayesianas com \emph{treewidth} limitado.
  \end{frame}

  \subsection{Por que estudar $k$-trees?}

  \begin{frame}
    \frametitle{Por que estudar \emph{$k$-trees}?}

    Há interesse considerável em desenvolver ferramentas eficientes para manipular \emph{$k$-trees}, porque \textbf{problemas NP-difíceis são resolvidos em tempo polinomial} em \emph{$k$-trees} e subgrafos de \emph{$k$-trees}.

    \vspace{1em}

    Alguns exemplos\footnote{\scriptsize Stefan Arnborg, Andrzej Proskurowski. Linear time algorithms for NP-Hard problems restricted to partial $k$-trees. \emph{Discrete Applied Mathematics}, 23:11--24, 1989.}:

    \begin{itemize}
      \item Encontrar tamanho máximo dos conjuntos independentes;
      \item Computar tamanho mínimo dos conjuntos dominantes;
      \item Calcular número cromático;
      \item Determinar se tem um ciclo hamiltoniano.
    \end{itemize}
  \end{frame}

  \subsection{Por que gerar $k$-trees?}

  \begin{frame}
    \frametitle{Por que gerar \emph{$k$-trees?}}

    Há muitas razões, como por exemplo para testar a eficácia de algoritmos aproximados.

    \vspace{1em}

    O problema que desperta nosso interesse é o \textbf{aprendizado de redes bayesianas}.
  \end{frame}

  \subsection{O que foi feito?}

  \begin{frame}
    \frametitle{O que foi feito?}

    \begin{itemize}
      \item Implementação do algoritmo de Caminiti \emph{et al.} (2010)\footnote{\scriptsize Severio Caminiti, Emanuele G. Fusco, Rossella Petreschi. Bijective linear time coding and decoding for $k$-trees. \emph{Theory of Computing Systems}, 46:284--300, 2010.} para \textbf{codificar \emph{$k$-trees} de forma bijetiva em tempo linear}.
      \item Implementação de algoritmo para \textbf{amostrar \emph{$k$-trees} uniformemente} e testes para comprovar seu funcionamento.
      \item Estudo sobre \textbf{aprendizado de redes bayesianas com \emph{treewidth} limitado} por meio da amostragem uniforme de \emph{$k$-trees} conforme artigo de Nie \emph{et al.} (2014)\footnote{\scriptsize Siqi Nie, Denis D. Mauá, Cassio P. de Campos, Qiang Ji. Advances in learning bayesian networks of bounded treewidth. \emph{CoRR}, abs/1406.1411, 2014.}.
      \item \textbf{Comparação entre métodos} para aprender redes bayesianas.
    \end{itemize}
  \end{frame}

  \subsection{Onde encontrar o trabalho?}

  \begin{frame}
    \frametitle{Onde encontrar o trabalho?}

    Código (desenvolvido em \emph{Go}\footnote{\scriptsize \url{https://golang.org/}}) e documentação: \url{https://github.com/tmadeira/tcc/}
  \end{frame}

  \section{Geração uniforme de $k$-trees}

  \subsection{O que são $k$-trees?}

  \begin{frame}
    \frametitle{Primeiramente, o que são \emph{$k$-trees}?}

    % TODO: definição, exemplos, construção
    % TODO: utilidade, treewidth
  \end{frame}

  \subsection{Codificação de $k$-trees}

  \begin{frame}
    \frametitle{Codificação de \emph{$k$-trees}}

    % TODO
  \end{frame}

  \subsection{Geração uniforme}

  \begin{frame}
    \frametitle{Geração uniforme de \emph{$k$-trees}}

    % TODO: motivação, algoritmo, implementação, Go
  \end{frame}

  \subsection{Testes}

  \begin{frame}
    \frametitle{Testes}

    % TODO: testes
  \end{frame}

  \section{Aprendizado de redes bayesianas}

  \subsection{Redes bayesianas}

  \begin{frame}
    \frametitle{Redes bayesianas}

    % TODO: definição, exemplos
  \end{frame}

  \subsection{Motivação}

  \begin{frame}
    \frametitle{Aprendizado de redes bayesianas}

    % TODO: Aprendizado de redes bayesianas: descrição
  \end{frame}

  \subsection{Aprendizado por amostragem de $k$-trees}

  \begin{frame}
    \frametitle{Aprendizado por amostragem de \emph{$k$-trees}}

    % TODO: descrição do algoritmo
  \end{frame}

  \subsection{Experimentos}

  \begin{frame}
    \frametitle{Experimentos}

    % TODO: resultados
  \end{frame}

  \section{Considerações finais}

  \subsection{Conclusões}

  \begin{frame}
    \frametitle{Conclusão}

    % TODO: escrever
  \end{frame}

  \subsection{Agradecimentos}

  \begin{frame}
    \frametitle{Agradecimentos}

    % TODO: escrever
  \end{frame}

  \begin{frame}
    \frametitle{Perguntas?}

    % TODO: escrever
  \end{frame}
\end{document}
