\documentclass[a4paper,12pt]{book}

\usepackage[utf8]{inputenc}
\usepackage[brazil]{babel}
\usepackage{courier}
\usepackage{emptypage}
\usepackage{hyperref}
\usepackage{natbib}
\usepackage{amsmath,amssymb,amsfonts,amsthm}
\usepackage{listings}
\usepackage{listings-golang}
\usepackage{tikz}
\usepackage{setspace}

\lstset{
  basicstyle=\footnotesize\ttfamily,
  breaklines=true,
  frame=L,
  language=Golang,
  numbers=left,
  tabsize=2
}

\newtheoremstyle{definition}
{\topsep}{\topsep}
{}{}
{\bfseries}{}
{ }
{\thmname{#1}~\thmnumber{#2}\thmnote{ (#3)}.}

\theoremstyle{definition}
\newtheorem{definition}{Definição}

\newtheoremstyle{algorithm}
{\topsep}{\topsep}
{}{}
{\scshape}{}
{ }
{\thmnote{#3}\\}

\theoremstyle{algorithm}
\newtheorem{algorithm}{Algoritmo}

\newtheoremstyle{step}
{\topsep}{\topsep}
{}{}
{\itshape}{}
{1em}
{\thmname{#1}~\thmnumber{#2}.}

\theoremstyle{step}
\newtheorem{step}{Passo}

\DeclareMathOperator*{\argmax}{arg\,max}

\frenchspacing
\linespread{1.5}

\title{Geração uniforme de \emph{k-trees} para aprendizado de redes bayesianas}
\author{Tiago Madeira}

\begin{document}

\frontmatter

  \thispagestyle{empty}
  \begin{center}
  \vspace*{1cm}
  Universidade de São Paulo\\
  Instituto de Matemática e Estatística\\
  Bacharelado em Ciência da Computação

  \vspace*{3cm}
  {\Large Tiago Madeira}

  \vspace{3cm}
  {
    \Large \bfseries
    Geração uniforme de \emph{k-trees} para aprendizado de redes bayesianas
  }

  \vspace{3cm}
  Supervisor: Prof. Dr. Denis Deratani Mauá

  \vspace{2cm}
  São Paulo

  Novembro de 2016
\end{center}

  \cleardoublepage\thispagestyle{empty}

  \pagenumbering{roman}
  \chapter*{Resumo}

Este trabalho consiste num estudo sobre amostragem uniforme de \emph{$k$-trees} e seu uso no aprendizado da estrutura de redes bayesianas com \emph{treewidth} limitado. Foi implementado um algoritmo para codificar e decodificar \emph{$k$-trees} de forma bijetiva em tempo linear. O trabalho mostra como aprender grafos acíclicos dirigidos cujo grafo moral é um subgrafo das \emph{$k$-trees} geradas. Experimentos comparam esse método para aprender estruturas com o estado da arte.

\vspace{1em}

\noindent \textbf{Palavras-chave:} \emph{$k$-trees}, codificação de grafos, amostragem uniforme, redes bayesianas, aprendizado de estrutura, \emph{treewidth} limitado


  \chapter*{Abstract}

$k$-trees are a generalization of trees to cyclical graphs that maintain their recursive aspect. There is considerable interest in developing efficient tools to manipulate this class of graphs because many NP-hard problems can be solved in polynomial time on $k$-trees and partial $k$-trees.

\vspace{1em}

\noindent This work is a study about uniform sampling of $k$-trees and its use to learn Bayesian networks with bounded treewidth. We have implemented algorithms for bijective linear time coding and decoding of $k$-trees. This work shows how to learn directed acyclic graphs whose moral graph is a subgraph of the generated $k$-trees. Experiments compare this method of structure learning with the state of the art.

\vspace{1em}

\noindent \textbf{Keywords:} $k$-trees, graph coding, uniform sampling, Bayesian networks, structure learning, bounded treewidth


  \tableofcontents
  \cleardoublepage

\mainmatter

  \chapter{Introdução}
\label{cap:introducao}

Em teoria dos grafos, \emph{$k$-trees} são consideradas uma generalização de árvores. Há interesse considerável em desenvolver ferramentas eficientes para manipular essa classe de grafos, porque todo grafo com \emph{treewidth} $k$ é um subgrafo de uma \emph{$k$-tree} e muitos problemas NP-completos podem ser resolvidos em tempo polinomial quando restritos a grafos com \emph{treewidth} limitada.

Com efeito, o artigo de Arnborg e Proskurowski \cite{arnborg} apresenta algoritmos para resolver em tempo linear problemas como, dado um grafo com \emph{treewidth} limitada:

\begin{itemize}
  \item Encontrar o tamanho máximo dos seus conjuntos independentes;
  \item Computar o tamanho mínimo dos seus conjuntos dominantes;
  \item Calcular seu número cromático; e
  \item Determinar se ele tem um ciclo hamiltoniano.
\end{itemize}

O problema que desperta nosso interesse em \emph{$k$-trees} é a inferência em redes bayesianas.

Uma rede bayesiana é um modelo probabilístico em grafo usado para raciocinar e tomar decisões em situações com incerteza através de técnicas de inteligência artificial e aprendizagem computacional. Ela representa uma distribuição de probabilidade multivariada num DAG (grafo acíclico dirigido) no qual os vértices correspondem às variáveis aleatórias do domínio e as arestas correspondem, intuitivamente, a influência de uma variável sobre outra.

Segundo Koller e Friedman \cite{koller}, a inferência em redes bayesianas em geral é NP-difícil; porém, se seu DAG possui \emph{treewidth} limitado, a inferência pode ser realizada em tempo polinomial. Daí a importância de aprender redes bayesianas que tenham \emph{treewidth} limitada.

A partir dessa motivação, este trabalho de conclusão de curso consistiu em estudar os conceitos de teoria dos grafos relacionados a \emph{$k$-trees} e implementar um algoritmo para gerar \emph{$k$-trees} de forma uniforme que possam ser usadas no aprendizado de redes bayesianas.

\section{Código desenvolvido}

As implementações deste trabalho foram realizadas na linguagem \emph{Go}\footnote{\emph{The Go Programming Language:} \url{https://golang.org/}}. \emph{Go} é uma linguagem de código aberto criada em 2007. Ela é compilada e usa tipagem estática como o C, mas por ser uma linguagem muito nova tem \emph{garbage collection} e recursos para programação concorrente.

Escolhemos \emph{Go} porque ela tem boa performance e é agradável de usar. Tem sistemas de pacotes ({\tt go get}), testes ({\tt go test}) e documentação (\emph{GoDoc}\footnote{\emph{GoDoc:} \url{https://godoc.org/}}) padronizados facilitando que os códigos sejam testados e reutilizados. Produz código limpo e padronizado (identação, espaçamento e outros detalhes de estilo são automatizados pela ferramenta {\tt gofmt} que vem com ela).

Todo o código desenvolvido neste trabalho está num repositório público no \emph{GitHub}\footnote{\emph{GitHub:} \url{https://github.com/}} cujo endereço é \url{https://github.com/tmadeira/tcc/}.

A documentação de todas as estruturas e funções declaradas no código está disponível em \url{https://godoc.org/github.com/tmadeira/tcc}.

Para baixar o código, rodar os testes e instalar os utilitários, recomenda-se usar as ferramentas da linguagem \emph{Go}:

\begin{lstlisting}
$ export ${GOPATH:=$HOME/go}
$ mkdir -p $GOPATH
$ go get github.com/tmadeira/tcc/...
$ go test -v github.com/tmadeira/tcc/...
$ go install github.com/tmadeira/tcc/examples/...
\end{lstlisting}

\section{Organização da monografia}

No capítulo \ref{cap:fundamentos}, apresentamos definições fundamentais de teoria dos grafos, teoria da probabilidade e redes bayesianas que o leitor deve conhecer para compreender o trabalho.

No capítulo \ref{cap:geracao}, apresentamos o problema de codificar \emph{$k$-trees}, discutimos os algoritmos lineares para codificar e decodificar \emph{$k$-trees} propostos no artigo \emph{``Bijective Linear Time Coding and Decoding for $k$-Trees''} \cite{caminiti}, explicamos como eles foram implementados neste trabalho para gerar \emph{$k$-trees} aleatórias uniformemente e apresentamos o resultado que obtivemos através de experimentos.

No capítulo \ref{cap:aprendizado}, explicamos como as \emph{$k$-trees} que geramos no capítulo \ref{cap:geracao} foram usadas para aprender redes bayesianas a partir do arcabouço desenvolvido no artigo \emph{``Advances in Learning Bayesian Networks of Bounded Treewidth''} \cite{maua}.

No capítulo \ref{cap:conclusao}, comparamos os resultados obtidos com o estado da arte e listamos as conclusões do trabalho.

  \cleardoublepage

  \chapter{Fundamentos}
\label{cap:fundamentos}

Neste capítulo, apresentamos definições fundamentais de teoria dos grafos, teoria da probabilidade e redes bayesianas que o leitor deve conhecer para compreender o trabalho.

Outras definições mais específicas, como as utilizadas para construir o algoritmo para codificar e decodificar \emph{$k$-trees}, estão localizadas nos capítulos subsequentes.

Partimos do pressuposto de que o leitor conhece notações básicas de conjuntos.

\section{Grafos}

Nesta seção apresentamos de forma breve apenas os conceitos de teoria dos grafos necessários para a compreensão deste trabalho. Mais detalhes podem ser encontrados no livro de Bondy e Murty \cite{bondy}, que foi utilizado como referência.

\begin{definition}[grafo]
  Um \textbf{grafo} é um par ordenado $G = (V, E)$. Os elementos de $V$ são chamados de \textbf{vértices} de $G$. Os elementos de $E$ são chamados de \textbf{arestas} de $G$ e consistem em pares (não-ordenados) de vértices distintos\footnote{A rigor, por causa da palavra ``distintos'', essa é a definição do que a literatura costuma chamar de \emph{grafo simples}. Tal definição é utilizada porque neste trabalho não temos interesse em grafos que possuam arestas $(u, v)$ com $u=v$.}. Dados $u, v \in V$, se $(u, v) \in E$ dizemos que $u$ e $v$ são \textbf{adjacentes} em $G$.

  A figura \ref{fig:grafo}(a) mostra como costuma ser representado um grafo. Os vértices são representados pelos círculos e as arestas são representadas pela ligação entre eles. Se $(u, v) \in E$, há uma linha ligando os vértices $u$ e $v$.

  Dado um grafo $G = (V, E)$, o número de arestas que incide num determinado vértice $v \in V$ é chamado de \textbf{grau} do vértice $v$.
\end{definition}

Há diferentes estruturas de dados que podem ser usadas para representar um grafo na memória do computador. Uma das mais comuns, que usamos nas implementações deste trabalho, é a \textbf{lista de adjacência}. Suponha, sem perda de generalidade, que os vértices de um grafo $G$ sejam representados por inteiros de $0$ a $|V|-1$. Então, a representação desse grafo consiste em um vetor de listas {\tt Adj}. A lista {\tt Adj[i]} contém os vértices adjacentes ao vértice de rótulo $i$ (para todo $i \in [0, |V|)$).

\begin{definition}[grafo dirigido]
  Um grafo $G = (V, E)$ é dito \textbf{dirigido} se $E$ consiste em pares \emph{ordenados} de vértices. Se $(a, b) \in E$, dizemos que $a$ aponta para $b$ ou que há uma aresta de $a$ para $b$.

  A figura \ref{fig:grafo}(b) mostra como costuma ser representado um grafo dirigido. Como o conjunto de arestas consiste em pares ordenados, elas são representadas por setas. Se $(u, v) \in E$, então a seta aponta de $u$ para $v$.
\end{definition}

\begin{definition}[grafo completo]
  Um grafo $G = (V, E)$ é dito \textbf{completo} se $(u, v) \in E$ para todo $u, v \in V, u \neq v$. Um grafo completo com $n$ vértices é geralmente denotado $K_n$.

  Na figura \ref{fig:grafo}(c), a representação de um grafo completo com $4$ vértices.
\end{definition}

\begin{figure}
  \begin{minipage}{0.3333\textwidth}
    \centering
    \begin{tikzpicture}
        [scale=.6,auto=left,every node/.style={draw, circle, inner sep = 0pt, minimum width = 0.72cm}]
      \node (n1) at (1,1) {1};
      \node (n2) at (4,4) {2};
      \node (n3) at (1,4) {3};
      \node (n4) at (4,1) {4};

      \foreach \from/\to in {n1/n3, n1/n4, n2/n3, n3/n4}
        \draw (\from) edge (\to);
    \end{tikzpicture}

    (a)
  \end{minipage}\begin{minipage}{0.3333\textwidth}
    \centering
    \begin{tikzpicture}
        [scale=.6,auto=left,every node/.style={draw, circle, inner sep = 0pt, minimum width = 0.72cm}]
      \node (n1) at (1,1) {1};
      \node (n2) at (4,4) {2};
      \node (n3) at (1,4) {3};
      \node (n4) at (4,1) {4};

      \foreach \from/\to in {n1/n3, n1/n4, n2/n3}
        \draw (\from) edge[->] (\to);
    \end{tikzpicture}

    (b)
  \end{minipage}\begin{minipage}{0.3333\textwidth}
    \centering
    \begin{tikzpicture}
        [scale=.6,auto=left,every node/.style={draw, circle, inner sep = 0pt, minimum width = 0.72cm}]
      \node (n1) at (1,1) {1};
      \node (n2) at (4,4) {2};
      \node (n3) at (1,4) {3};
      \node (n4) at (4,1) {4};

      \foreach \from/\to in {n1/n2, n1/n3, n1/n4, n2/n3, n2/n4, n3/n4}
        \draw (\from) edge (\to);
    \end{tikzpicture}

    (c)
  \end{minipage}

  \caption{
    \textbf{(a)} Representação do grafo $G = (V_G, E_G)$ com $V_G = \{1, 2, 3, 4\}$ e $E_G = \{(1, 3), (1, 4), (2, 3), (3, 4)\}$.
    \textbf{(b)} Representação do grafo dirigido $D = (V_D, E_D)$ com $V_D = \{1, 2, 3, 4\}$ e $E_D = \{(1, 3), (1, 4), (2, 3)\}$.
    \textbf{(c)} Representação do $K_4$, o grafo completo com $4$ vértices.
  }
  \label{fig:grafo}
\end{figure}

\begin{definition}[subgrafo]
  Um grafo $F = (V_F, E_F)$ é chamado de \textbf{subgrafo} de $G = (V_G, E_G)$ se $V_F \subseteq V_G$ e $E_F \subseteq E_G$.
\end{definition}

\begin{definition}[subgrafo induzido]
  Dado um grafo $G = (V, E)$ e um subconjunto $V'$ de $V$, o subgrafo de $G$ \textbf{induzido} por $V'$, $G' = (V', E')$, é o grafo formado pelos vértices $V' \subseteq V$ e arestas que só contém elementos de $V'$, ou seja, $E' = \{(u, v) \in E \ | \  u, v \in V'\}$.
\end{definition}

\begin{definition}[caminho]
  Dado um grafo $G = (V, E)$, um \textbf{caminho} em $G$ é um subgrafo de $G$ cujos vértices podem ser arranjados numa sequência linear de forma que dois vértices são adjacentes se eles são consecutivos na sequência e não-adjacentes caso contrário. Se $u, v \in V$ pertencem a um caminho $P$, dizemos que eles estão conectados pelo caminho $P$.
\end{definition}

\begin{definition}[distância]
  Dado um grafo $G = (V, E)$ e dois vértices $(u, v) \in V$, a \textbf{distância} entre $u$ e $v$ é o número de arestas num menor caminho que os conecte.
\end{definition}

\begin{definition}[ciclo]
  Dado um grafo $G = (V, E)$, um \textbf{ciclo} em $G$ é um caminho formado por vértices $x_1, \cdots, x_k \in V$ onde $x_1 = x_k$.
\end{definition}

\begin{definition}[DAG]
  Um grafo $G = (V, E)$ é chamado de \textbf{DAG} (do inglês \emph{directed acyclic graph}: grafo dirigido acíclico) se ele é dirigido e não possui ciclos.
\end{definition}

\begin{definition}[árvore]
  Dado um grafo $G = (V, E)$, dizemos que ele é uma \textbf{árvore} se cada dois vértices $u, v \in V$ são conectados por exatamente um caminho.
\end{definition}

Dada uma árvore $T = (V, E)$, os vértices em $V$ que tem grau $1$ são chamados de \textbf{folhas}.

Dada uma árvore $T = (V, E)$, às vezes é conveniente destacar um vértice $r \in V$ e chamá-lo de \textbf{raiz} da árvore $T$. Chamamos o par formado pela árvore $T$ e pela raiz $r \in V$ de \textbf{árvore enraizada}.

\begin{definition}[$k$-clique]
  Seja $G = (V, E)$ um grafo. Um \textbf{$k$-clique} é um subconjunto dos vértices, $C \subseteq V$, tal que $(u, v) \in E \ \forall \ u, v \in C, u \neq v$ (ou seja, tal que o subgrafo induzido por $C$ é completo).
\end{definition}

\subsection{\emph{$k$-trees}}

\begin{definition}[\emph{$k$-tree}]
  \label{def:ktree}
  \cite{harary} Uma \textbf{\emph{$k$-tree}} é definida da seguinte forma recursiva:

  \begin{enumerate}
    \item Um grafo completo com $k$ vértices é uma \emph{$k$-tree}.
    \item Se $T_k' = (V, E)$ é uma \emph{$k$-tree}, $K \subseteq V$ é um $k$-clique e $v \not \in V$, então $T_k = (V \cup \{v\}, E \cup \{(v,x) \ | \  x \in K\})$ é uma \emph{$k$-tree}.
  \end{enumerate}

  Na figura \ref{fig:ktree}(a), um exemplo de \emph{$k$-tree} com $k = 1$ (ou seja, uma árvore comum) e $n = 4$ vértices rotulados com inteiros em $[1, 4]$; na figura \ref{fig:ktree}(b), um exemplo de \emph{$k$-tree} com $k = 2$ e $n = 5$ vértices rotulados com inteiros em $[1, 5]$; na figura \ref{fig:ktree}(c), um exemplo de \emph{$k$-tree} com $k = 3$ e $n = 5$ vértices rotulados em $[1, 5]$.

  \begin{figure}
    \begin{minipage}{0.3333\textwidth}
      \centering
      \begin{tikzpicture}
          [scale=.6,auto=left,every node/.style={draw, circle, inner sep = 0pt, minimum width = 0.72cm}]
        \node (n1) at (3,8) {1};
        \node (n2) at (1.5,6) {2};
        \node (n3) at (4.5,6) {3};
        \node (n4) at (2.25,4) {4};

        \foreach \from/\to in {n1/n2, n1/n3, n2/n4}
          \draw (\from) edge (\to);
      \end{tikzpicture}

      (a)
    \end{minipage}\begin{minipage}{0.3333\textwidth}
      \centering
      \begin{tikzpicture}
          [scale=.6,auto=left,every node/.style={draw, circle, inner sep = 0pt, minimum width = 0.72cm}]
        \node (n1) at (3,8) {1};
        \node (n2) at (5,8) {2};
        \node (n3) at (2.5,6) {3};
        \node (n4) at (6,6) {4};
        \node (n5) at (3.5,4) {5};

        \foreach \from/\to in {n1/n2, n3/n1, n3/n2, n4/n1, n4/n2, n5/n3, n5/n2}
          \draw (\from) edge (\to);
      \end{tikzpicture}

      (b)
    \end{minipage}\begin{minipage}{0.3333\textwidth}
      \centering
      \begin{tikzpicture}
          [scale=.6,auto=left,every node/.style={draw, circle, inner sep = 0pt, minimum width = 0.72cm}]
        \node (n1) at (3,8) {1};
        \node (n2) at (5,8) {2};
        \node (n3) at (7,8) {3};
        \node (n4) at (4,6) {4};
        \node (n5) at (5,4) {5};

        \foreach \from/\to in {n1/n2, n2/n3, n1/n4, n2/n4, n3/n4, n4/n5, n3/n5}
          \draw (\from) edge (\to);

        \draw (n1) edge [bend left=40] (n3);
        \draw (n1) edge [bend right=40] (n5);
      \end{tikzpicture}

      (c)
    \end{minipage}

    \caption{
      \textbf{(a)} Uma \emph{$1$-tree} (ou seja, uma árvore comum) com $4$ vértices.
      \textbf{(b)} Uma \emph{$2$-tree} com $5$ vértices.
      \textbf{(c)} Uma \emph{$3$-tree} com $5$ vértices.
    }
    \label{fig:ktree}
  \end{figure}
\end{definition}

\begin{definition}[\emph{$k$-tree} enraizada]
  \cite{caminiti} Uma \textbf{\emph{$k$-tree} enraizada} é uma \emph{$k$-tree} com um $k$-clique destacado $R = \{r_1, r_2, \cdots, r_k\}$ que é chamado de \emph{raiz} da \emph{$k$-tree} enraizada.

  Na figura \ref{fig:rootedktree}(a), um exemplo de uma \emph{$k$-tree} com $k = 3$ e $n = 11$ vértices rotulados com inteiros em $[1, 11]$. Na figura \ref{fig:rootedktree}(b), a mesma \emph{$k$-tree}, dessa vez enraizada no $3$-clique $R = \{2, 3, 9\}$.

  \begin{figure}
    \begin{minipage}{0.5\textwidth}
      \centering
      \begin{tikzpicture}
          [scale=.6,auto=left,every node/.style={draw, circle, inner sep = 0pt, minimum width = 0.72cm}]
        \node (n10) at (1,9) {10};
        \node (n2) at (2.5,7) {2};
        \node (n1) at (1,4) {1};
        \node (n5) at (3,2.75) {5};
        \node (n7) at (2,1) {7};
        \node (n9) at (4.5,9.5) {9};
        \node (n6) at (4,5) {6};
        \node (n4) at (6,10.5) {4};
        \node (n3) at (8.5,6.5) {3};
        \node (n8) at (8,4.5) {8};
        \node (n11) at (9,9) {11};

        \foreach \from/\to in {n1/n2, n1/n5, n1/n7, n1/n8, n2/n3, n2/n5, n2/n6, n2/n8, n2/n9, n2/n10, n2/n11, n3/n4, n3/n5, n3/n8, n3/n9, n3/n10, n3/n11, n4/n9, n4/n11, n5/n7, n5/n8, n6/n8, n6/n9, n7/n8, n8/n9, n9/n10, n9/n11}
          \draw (\from) edge (\to);
      \end{tikzpicture}

      (a)
    \end{minipage}\begin{minipage}{0.5\textwidth}
      \centering
      \begin{tikzpicture}
          [scale=.6,auto=left,every node/.style={draw, circle, inner sep = 0pt, minimum width = 0.72cm}]
        \node (n10) at (1,8.25) {10};
        \node[fill=gray!30] (n2) at (1.5,10.5) {2};
        \node (n1) at (2,3.75) {1};
        \node (n5) at (3,6) {5};
        \node (n7) at (3.5,1.5) {7};
        \node[fill=gray!30] (n9) at (4.5,10.5) {9};
        \node (n6) at (6,6) {6};
        \node (n4) at (9,4) {4};
        \node[fill=gray!30] (n3) at (7.5,10.5) {3};
        \node (n8) at (4.5,8.25) {8};
        \node (n11) at (8,8.25) {11};

        \foreach \from/\to in {n1/n5, n1/n7, n2/n5, n2/n8, n2/n9, n2/n10, n2/n11, n3/n8, n3/n9, n3/n10, n3/n11, n4/n9, n4/n11, n5/n7, n5/n8, n6/n8, n6/n9, n8/n9, n9/n10, n9/n11}
          \draw (\from) edge (\to);

        \draw (n1) edge [bend right=20] (n8);
        \draw (n1) edge [bend left=50] (n2);
        \draw (n2) edge [bend left] (n3);
        \draw (n2) edge [bend right=20] (n6);
        \draw (n3) edge [bend left] (n4);
        \draw (n3) edge [bend left=20] (n5);
        \draw (n7) edge [bend right=20] (n8);
      \end{tikzpicture}

      (b)
    \end{minipage}

    \caption{
      \textbf{(a)} Uma \emph{$3$-tree} $T_3$ com 11 vértices.
      \textbf{(b)} A mesma \emph{$3$-tree} ($T_3$) enraizada no $3$-clique $\{2, 3, 9\}$.
    }
    \label{fig:rootedktree}
  \end{figure}
\end{definition}

\begin{definition}[\emph{partial $k$-tree}]
  \cite{bodlaender} Um subgrafo de uma \emph{$k$-tree} é chamado de \textbf{\emph{partial $k$-tree}}. Um grafo é uma \emph{partial $k$-tree} se e só se ele tem \emph{treewidth} menor ou igual a $k$. % TODO: definir treewidth?
\end{definition}

\section{Probabilidade}

Nesta seção apresentamos de forma sintética alguns conceitos de teoria da probabilidade necessários para a compreensão deste trabalho. Mais detalhes podem ser encontrados no livro de Koller e Friedman \cite{koller}, que foi utilizado como referência.

Um \textbf{experimento aleatório} é um fenômeno que possui resultado imprevisível. Por exemplo, o lançamento de um dado de seis faces é um experimento aleatório.

O \textbf{espaço amostral} de um experimento aleatório, geralmente denotado $\Omega$, é o conjunto de todos os resultados possíveis do experimento. Por exemplo, o espaço amostral do lançamento de um dado de seis faces é $\{1, 2, 3, 4, 5, 6\}$.

\textbf{Eventos} são subconjuntos de um espaço amostral para os quais pretendemos atribuir probabilidades. Por exemplo, no lançamento de um dado de seis faces o evento $\{6\}$ corresponde ao caso em que o lançamento resulta em $6$ e o evento $\{1, 2\}$ corresponde ao caso em que o lançamento resulta em $1$ ou $2$.

A teoria da probabilidade requer que o espaço dos eventos $S \subseteq \Omega$ satisfaça três propriedades:

\begin{itemize}
  \item Deve conter o \textbf{evento vazio} $\emptyset$ e o \textbf{evento trivial} $\Omega$.
  \item Se $\alpha, \beta \in S$, então $\alpha \cup \beta \in S$.
  \item Se $\alpha \in S$, então $\Omega \setminus \alpha \in S$.
\end{itemize}

O requisito de que o espaço dos eventos é fechado sob a união e o complemento implica que ele também seja fechado sob outras operações booleanas como interseção e diferença.

\begin{definition}[distribuição de probabilidade]
  Seja $\Omega$ um espaço amostral e $S$ um espaço de eventos. Uma \textbf{distribuição de probabilidade} $P$ sobre $(\Omega, S)$ é um mapeamento dos eventos em $S$ para valores reais que satisfaz as seguintes condições:

  \begin{itemize}
    \item Probabilidades são não-negativas, ou seja, $P(\alpha) \geq 0$ para todo $\alpha \in S$.
    \item O evento trivial tem a maior probabilidade possível, ou seja, $P(\Omega) = 1$.
    \item A probabilidade de que um de dois eventos disjuntos ocorra é a soma das probabilidades de cada evento, ou seja, se $\alpha, \beta \in S$ e $\alpha \cup \beta = \emptyset$, então $P(\alpha \cup \beta) = P(\alpha) + P(\beta)$.
  \end{itemize}

  Essas condições implicam em outras. Em particular, vale destacar que $P(\emptyset) = 0$ e $P(\alpha \cup \beta) = P(\alpha) + P(\beta) - P(\alpha \cap \beta)$.
\end{definition}

\begin{definition}[probabilidade condicional]
  Seja $P$ uma distribuição de probabilidade sobre $(\Omega, S)$ e sejam $\alpha, \beta \in S$. A \textbf{probabilidade condicional} de $\beta$ dado $\alpha$, $P(\beta | \alpha)$, é definida como:

  $$P(\beta | \alpha) = \frac{P(\alpha \cap \beta)}{P(\alpha)}$$
\end{definition}

Duas consequências da definição da probabilidade condicional são as fórmulas que chamamos de \textbf{regra da cadeia} e \textbf{regra de Bayes}:

\begin{description}
  \item[Regra da Cadeia.] Se $\alpha_1, \cdots, \alpha_k \in S$, então

    $$P(\alpha_1 \cap \cdots \alpha_k) = P(\alpha_1) P(\alpha_2 | \alpha_1) \cdots P(\alpha_k | \alpha_1 \cap \cdots \cap \alpha_{k-1})$$

  \item[Regra de Bayes.] Se $\alpha, \beta \in S$, então

    $$P(\alpha | \beta) = \frac{P(\beta | \alpha) P(\alpha)}{P(\beta)}$$
\end{description}

\vspace{2em}

A continuar.
% TODO: variável aleatória
% TODO: probabilidade conjunta
% TODO: independência

\subsection{Redes bayesianas}

Redes bayesianas são modelos probabilísticos gráficos que representam distribuições de probabilidade conjunta e são usados para raciocinar em situações com incerteza. Formalmente:

\begin{definition}[rede bayesiana]
  \cite{maua}
  Seja $N = \{ 1, \cdots, n \}$ e seja $X = \{X_i : i \in N\}$ um conjunto de variáveis aleatórias $X_i$ tomando valores em conjuntos finitos $\mathcal{X}_i$.

  Uma \textbf{rede bayesiana} é uma tripla $(X, G, \theta)$, onde $G = (V, E)$ é um DAG cujos vértices correspondem a variáveis em $X$ e $\theta = \{\theta_i(x_i, x_{\phi_i})\}$ é um conjunto de parâmetros numéricos especificando valores de probabilidade condicional $\phi_i(x_i, x_{\phi_i}) = P(x_i | x_{\phi_i})$ para todo vértice $i \in V$, valor $x_i \in X_i$ e atribuição $x_{\phi_i}$ para os pais $\phi_i$ de $X_i$ (em $G$).
\end{definition}

\vspace{2em}

A continuar.
% TODO: onde falar de aprendizagem e inferência?

  \cleardoublepage

  \chapter{Geração uniforme de \emph{$k$-trees}}
\label{cap:geracao}

O problema de gerar \emph{$k$-trees} está intimamente relacionado ao problema de codificá-las e decodificá-las. De fato, se há uma codificação bijetiva que associa \emph{$k$-trees} a \emph{strings}, basta gerar \emph{strings} uniformemente aleatórias para gerar \emph{$k$-trees} uniformemente aleatórias.

Neste capítulo, apresentamos o problema de codificar \emph{$k$-trees}, discutimos a solução linear para codificar e decodificar \emph{$k$-trees} de forma bijetiva proposta por Caminiti \emph{et al.} \cite{caminiti}, explicamos como ela foi implementada neste trabalho para gerar \emph{$k$-trees} aleatórias e mostramos os resultados obtidos.

\section{Codificando árvores e \emph{$k$-trees}}

\emph{$k$-trees} \cite{harary} são consideradas uma generalização de árvores. Há interesse considerável em desenvolver ferramentas eficientes para manipular essa classe de grafos, porque todo grafo com \emph{treewidth} $k$ é um subgrafo de uma \emph{$k$-tree} e muitos problemas NP-completos podem ser resolvidos em tempo polinomial quando restritos a grafos com \emph{treewidth} limitado, como destacado no capítulo \ref{cap:introducao} deste trabalho.

\vspace{2em}

O problema de codificar árvores já foi amplamente estudado na literatura. Como destaca Caminiti \emph{et al.} \cite{caminiti}:

\begin{quotation}
  Codificar árvores rotuladas por meio de \emph{strings} de rótulos de vértices é uma alternativa interessante à representação usual de estruturas de dados de árvore na memória e tem muitas aplicações práticas (por exemplo, algoritmos evolucionários sobre árvores, geração aleatória de árvores, compressão de dados e computação do volume de floresta de grafos). Diversos códigos bijetivos diferentes que realizam associações entre árvores rotuladas e \emph{strings} de rótulos foram introduzidas. De um ponto de vista algorítmico, o problema foi cuidadosamente investigado e algoritmos ótimos de codificação e decodificação desses códigos são conhecidos.
\end{quotation}

Em 1889, Cayley \cite{cayley} demonstrou que para um conjunto de $n$ vértices distintos existem $n^{n-2}$ árvores possíveis. Desde lá, foram criados vários códigos para associar \emph{strings} e árvores.

Um dos mais conhecidos é o código de Prüfer \cite{prufer}, que surgiu em 1918 e é bijetivo, associando cada árvore (rotulada) de $n$ vértices a uma lista distinta de comprimento $n-2$ no alfabeto dos rótulos da árvore.

Codificar uma árvore usando o código de Prüfer é trivial: basta remover iterativamente as folhas da árvore até que apenas dois vértices sobrem, escolhendo sempre a folha de menor rótulo. Quando uma folha é removida, adiciona-se ao código o rótulo do seu vizinho.

A figura \ref{fig:prufer} exemplifica a codificação de Prüfer mostrando uma árvore cujo o código resultante do algoritmo é $\{4, 4, 4, 5\}$.

\begin{figure}
  \centering
  \begin{tikzpicture}
      [scale=.6,auto=left,every node/.style={draw, circle, inner sep = 0pt, minimum width = 0.72cm}]
    \node (n1) at (1,8) {1};
    \node (n2) at (4,8) {2};
    \node (n3) at (6,6) {3};
    \node (n4) at (3,5) {4};
    \node (n5) at (3.5,3) {5};
    \node (n6) at (0,2) {6};

    \foreach \from/\to in {n1/n4, n2/n4, n3/n4, n4/n5, n5/n6}
      \draw (\from) edge (\to);
  \end{tikzpicture}

  \caption{A árvore rotulada equivalente ao código de Prüfer $\{4, 4, 4, 5\}$.}
  \label{fig:prufer}
\end{figure}

\vspace{2em}

Há estudos sobre a codificação de \emph{$k$-trees} há pelo menos quatro décadas. Em 1970, Rényi e Rényi apresentaram uma codificação redundante (ou seja, não bijetiva) para um subconjunto de \emph{$k$-trees} rotuladas que chamamos de \emph{$k$-trees} de Rényi e que são definidas como segue:

\begin{definition}[\emph{$k$-tree} de Rényi]
  \cite{renyi} Uma \emph{$k$-tree} de Rényi $R_k$ é uma \emph{$k$-tree} enraizada com $n$ vértices rotulados em $[1, n]$ e raiz $R = \{n-k+1, n-k+2, \cdots, n\}$.
\end{definition}

Entretanto apenas em 2008 surgiu um código bijetivo para \emph{$k$-trees} com algoritmos lineares de codificação e decodificação. Foram esses algoritmos, propostos por Caminiti \emph{et al.} \cite{caminiti}, que implementamos neste trabalho.

\section{A solução de Caminiti \emph{et al.}}
\label{sec:caminiti}

O artigo \emph{``Bijective Linear Time Coding and Decoding for $k$-Trees''} \cite{caminiti} apresenta um código bijetivo para \emph{$k$-trees} rotuladas, juntamente a algoritmos lineares para realizar a codificação e a decodificação.

O código é formado por uma permutação de tamanho $k$ e uma generalização do \emph{Dandelion Code} \cite{egecioglu}, que consiste em $n-k-2$ pares (onde $n$ é o número de vértices) definidos no conjunto $\{ ( 0, \varepsilon ) \} \cup ([1,n-k] \times [1,k])$. Portanto, dizemos que a codificação das \emph{$k$-trees} associa elementos em $\mathcal{T}^n_k$ (conjunto das \emph{$k$-trees} com $n$ vértices) com elementos em:

$$
\mathcal{A}^n_k = { [1,n] \choose k } \times (\{ ( 0, \varepsilon ) \} \cup ([1,n-k] \times [1,k]))^{n-k-2}
$$

Caminiti \emph{et al.} \cite{caminiti} mostra que a estrutura dessas \emph{strings} que o \emph{Dandelion Code} gera é essencial para garantir a bijetividade.

\vspace{2em}

Os algoritmos consistem em uma série de transformações. Para compreendê-los, é necessário definir esqueleto de uma \emph{$k$-tree} enraizada e árvore característica:

\begin{definition}[esqueleto de uma \emph{$k$-tree} enraizada]
  \label{def:skeleton}
  \cite{caminiti} O esqueleto de uma \emph{$k$-tree} enraizada $T_k$ com raiz $R$, denotado por $S(T_k, R)$, é definido da seguinte forma recursiva:

  \begin{enumerate}
    \item Se $T_k$ é apenas o $k$-clique $R$, seu esqueleto é uma árvore com um único vértice $R$.
    \item Dada uma \emph{$k$-tree} enraizada $T_k$ com raiz $R$, obtida por $T_k'$ enraizada em $R$ através da adição de um novo vértice $v$ conectado a um $k$-clique $K$ (ver definição \ref{def:ktree}), seu esqueleto $S(T_k, R)$ é obtido adicionando a $S(T_k', R)$ um novo vértice $X = \{v\} \cup K$ e uma nova aresta $(X, Y)$, onde $Y$ é o vértice de $S(T_k', R)$ que contém $K$ com uma distância mínima da raiz. Chamamos $Y$ de pai de $X$.
  \end{enumerate}
\end{definition}

\begin{definition}[árvore característica]
  \label{def:chartree}
  \cite{caminiti} A árvore característica $T(T_k, R)$ de uma \emph{$k$-tree} enraizada $T_k$ com raiz $R$ é obtida rotulando os vértices e arestas de $S(T_k, R)$ da seguinte forma:

  \begin{enumerate}
    \item O vértice $R$ é rotulado $0$ e cada vértice $\{v\} \cup K$ é rotulado $v$;
    \item Cada aresta do vértice $\{v\} \cup K$ ao seu pai $\{v'\} \cup K'$ é rotulada com o índice do vértice em $K'$ (visualizando-o como um conjunto ordenado) que não aparece em $K$. Quando o pai é $R$ a aresta é rotulada $\varepsilon$.
  \end{enumerate}

  Note que a existência de um único vértice em $K' \setminus K$ é garantida pela definição \ref{def:skeleton}. De fato, $v'$ precisa aparecer em $K$, caso contrário $K' = K$ e o pai de $\{v'\} \cup K'$ contém $K$. Isso contradiz o fato de que cada vértice em $S(T_k, R)$ é ligado à distância mínima da raiz.
\end{definition}

A figura \ref{fig:transformation} mostra uma \emph{$k$-tree} de Rényi com $11$ vértices, seu esqueleto e sua árvore característica. O \emph{Dandelion Code} generalizado correspondente a essa árvore é $[(0, \varepsilon), (2, 0), (8, 2), (8, 1), (1, 2), (5, 2)]$. A forma como codificamos e decodificamos árvores características usando esse código será vista a seguir, nos algoritmos de codificação e decodificação.

\begin{figure}
  \begin{minipage}{0.3\textwidth}
    \centering
    \scalebox{0.75}{
      \begin{tikzpicture}
          [scale=.5,auto=left,every node/.style={draw, circle, inner sep = 0pt, minimum width = 0.7cm}]
        \node (n10) at (1,9) {3};
        \node[fill=gray!30] (n2) at (1.5,12) {9};
        \node (n1) at (2,3) {1};
        \node (n5) at (3,6) {5};
        \node (n7) at (3.5,0) {7};
        \node[fill=gray!30] (n9) at (4.5,12) {11};
        \node (n6) at (6,6) {6};
        \node (n4) at (9,3) {4};
        \node[fill=gray!30] (n3) at (7.5,12) {10};
        \node (n8) at (4.5,9) {8};
        \node (n11) at (8,9) {2};

        \foreach \from/\to in {n1/n5, n1/n7, n2/n5, n2/n8, n2/n9, n2/n10, n2/n11, n3/n8, n3/n9, n3/n10, n3/n11, n4/n9, n4/n11, n5/n7, n5/n8, n6/n8, n6/n9, n8/n9, n9/n10, n9/n11}
          \draw (\from) edge (\to);

        \draw (n1) edge [bend right=20] (n8);
        \draw (n1) edge [bend left=50] (n2);
        \draw (n2) edge [bend left] (n3);
        \draw (n2) edge [bend right=20] (n6);
        \draw (n3) edge [bend left] (n4);
        \draw (n3) edge [bend left=20] (n5);
        \draw (n7) edge [bend right=20] (n8);
      \end{tikzpicture}
    }

    (a)
  \end{minipage}\begin{minipage}{0.5\textwidth}
    \centering
    \scalebox{0.64}{\setstretch{1.0}
      \begin{tikzpicture}
          [scale=.6,auto=left,every node/.style={draw, rectangle, rounded corners = .1cm, inner sep = 0pt, minimum height = 1.25cm, minimum width = 2.25cm, align = center}]
        \node[fill=gray!30] (n0) at (5,12) {$\{9, 10, 11\}$};
        \node (n3) at (0.5,9) {$\{3\} \cup$ \\ $\{9, 10, 11\}$};
        \node (n8) at (5,9) {$\{8\} \cup$ \\ $\{9, 10, 11\}$};
        \node (n2) at (9.5,9) {$\{2\} \cup$ \\ $\{9, 10, 11\}$};
        \node (n5) at (2,6) {$\{5\} \cup$ \\ $\{8, 9, 10\}$};
        \node (n6) at (6.5,6) {$\{6\} \cup$ \\ $\{8, 9, 11\}$};
        \node (n4) at (11,6) {$\{4\} \cup$ \\ $\{2, 10, 11\}$};
        \node (n1) at (2,3) {$\{1\} \cup$ \\ $\{5, 8, 9\}$};
        \node (n7) at (2,0) {$\{5\} \cup$ \\ $\{1, 5, 8\}$};

      \foreach \from/\to in {n0/n3, n0/n8, n0/n2, n8/n5, n8/n6, n2/n4, n5/n1, n1/n7}
        \draw (\from) edge (\to);
      \end{tikzpicture}
    }

    (b)
  \end{minipage}\begin{minipage}{0.2\textwidth}
    \centering
    \begin{tikzpicture}
        [scale=.6,auto=left,every node/.style={draw, circle, inner sep = 0pt, minimum width = 0.65cm}]
      \node[fill=gray!30] (n0) at (5,8) {0};
      \node (n3) at (3.5,6) {3};
      \node (n8) at (5,6) {8};
      \node (n2) at (6.5,6) {2};
      \node (n5) at (3.75,4) {5};
      \node (n6) at (5.25,4) {6};
      \node (n4) at (6.75,4) {4};
      \node (n1) at (4,2) {1};
      \node (n7) at (4,0) {7};

      \foreach \from/\to/\elab in {n0/n3/$\varepsilon$, n0/n8/$\varepsilon$, n0/n2/$\varepsilon$, n8/n5/3, n8/n6/2, n2/n4/1, n5/n1/3, n1/n7/3}
        \draw (\from) -- (\to) node [draw=none, minimum width = 0.3cm, midway] {\scriptsize \elab};
    \end{tikzpicture}

    (c)
  \end{minipage}

  \caption{
    \textbf{(a)} Uma \emph{$3$-tree} de Rényi $R_3$ com 11 vértices e raiz $\{9, 10, 11\}$.
    \textbf{(b)} O esqueleto de $R_3$.
    \textbf{(c)} A árvore característica de $R_3$.
  }
  \label{fig:transformation}
\end{figure}

\subsection{Codificação}
\label{subsec:codificacao}

O algoritmo para codificar uma \emph{$k$-tree} rotulada consiste em cinco passos e tem complexidade $O(nk)$. Aqui apresentamos esse algoritmo indicando onde cada um dos passos pode ser encontrado na nossa implementação.

\begin{algorithm}[Algoritmo de codificação]
  \textbf{Entrada:} uma \emph{$k$-tree} $T_k$ com $n$ vértices\\
  \textbf{Saída:} um código $(Q, S)$ em $\mathcal{A}^n_k$

  \begin{enumerate}
    \item Identificar $Q$, o $k$-clique adjacente à folha de maior rótulo $l_M$ de $T_k$;
    \item Através de um processo de re-rotulação $\phi$ (computado a partir de $Q$ e detalhado a seguir), transformar $T_k$ numa \emph{$k$-tree} de Rényi $R_k$;
    \item Gerar a árvore característica $T$ para $R_k$;
    \item Computar o \emph{Dandelion Code} generalizado $S$ para $T$;
    \item Remover da \emph{string} obtida $S$ o par correspondente a $\phi(l_M)$.
  \end{enumerate}

  O algoritmo retorna o par $(Q, S)$ computado durante esse processo.

  \vspace{2em}

  Na nossa implementação, uma \emph{$k$-tree} (estrutura definida no pacote {\tt ktree}) é representada através de uma lista de adjacências ({\tt Adj}) e um inteiro $k$ ({\tt K}).

  O algoritmo de codificação é implementado pela função {\tt CodingAlgorithm} do pacote {\tt codec}. A seguir, detalhamos os cinco passos.

  \begin{step}
    Primeiramente precisamos encontrar $l_M$, a folha de $T_k$ com maior rótulo. Uma folha em uma \emph{$k$-tree} consiste em um vértice de grau $k$, portanto basta iterar na lista de adjacências em ordem decrescente nos rótulos até encontrar um vértice com grau $k$. Isso foi implementado na função {\tt FindLm}, localizada no pacote {\tt ktree}.

    Encontrado $l_M$, atribuímos a $Q$ a lista {\tt Adj[$l_M$]} (ver função {\tt GetQ} do pacote {\tt ktree}).
  \end{step}

  \begin{step}
    Queremos transformar $T_k$ numa \emph{$k$-tree} de Rényi enraizada em $Q$. Para isso, precisamos definir uma permutação que associe os vértices de $Q$ a $\{n-k+1, n-k+2, \cdots, n\}$. A função de permutação, que chamamos de $\phi$, é definida da seguinte forma:

    \begin{enumerate}
      \item Se $q_i$ é o $i$-ésimo menor vértice em $Q$, fazemos $\phi(q_i) = n-k+i$;
      \item Para cada $q \not \in Q \cup \{n-k+1, \cdots, n\}$, fazemos $\phi(q) = q$;
      \item O restante dos valores são usados para fechar os ciclos de permutação, ou seja, para cada $q \in \{n-k+1, \cdots, n\} \setminus Q$, fazemos $\phi(q) = i$ tal que $\phi^j(i) = q$ e $j$ é maximizado.
    \end{enumerate}

    Essa computação é implementada pela função {\tt ComputePhi} no pacote {\tt ktree}.

    Usamos a função $\phi$ para re-rotular os vértices de $T_k$, obtendo a \emph{$k$-tree} de Rényi $R_k$. A implementação desse processo foi realizada na função {\tt Relabel} do pacote {\tt ktree}.

    A figura \ref{fig:phi} mostra uma representação gráfica da função $\phi$ usada para re-rotular a \emph{$3$-tree} mostrada na figura \ref{fig:rootedktree} com $Q = \{2, 3, 9\}$ produzindo a \emph{$k$-tree} de Rényi mostrada na figura \ref{fig:transformation}(a).

    \begin{figure}
      \centering
      \begin{tikzpicture}
          [scale=.6,auto=left,every node/.style={circle, inner sep = 0pt, minimum width = 0.55cm}]
        \node (n2) at (1,0) {2};
        \node (n9) at (2.5,0) {9};
        \node (n11) at (4,0) {11};

        \node (n3) at (6,0) {3};
        \node (n10) at (7.5,0) {10};

        \node (n1) at (9.5,0) {1};

        \node (n4) at (11.5,0) {4};

        \node (n5) at (13.5,0) {5};

        \node (n6) at (15.5,0) {6};

        \node (n7) at (17.5,0) {7};

        \node (n8) at (19.5,0) {8};

        \foreach \from/\to in {n2/n9, n9/n11, n11/n2, n3/n10, n10/n3}
          \draw (\from) edge[-triangle 90,bend right] (\to);

        \foreach \vv in {n1, n4, n5, n6, n7, n8}
          \draw (\vv) edge[loop,->,>=triangle 90] (\vv);
      \end{tikzpicture}

      \caption{Representação gráfica da função $\phi$ computada para a \emph{$3$-tree} mostrada na figura \ref{fig:rootedktree}.}
      \label{fig:phi}
    \end{figure}
  \end{step}

  \begin{step}
    As definições \ref{def:skeleton} e \ref{def:chartree} sugerem algoritmos triviais para gerar a árvore característica $T$ para a \emph{$k$-tree} de Rényi $R_k$ obtida no passo anterior por meio do seu esqueleto (o processo visto na figura \ref{fig:transformation}).

    Para garantir tempo linear, no entanto, o artigo de Caminiti \emph{et al.} \cite{caminiti} sugere evitar a construção explícita do esqueleto $S(R_k)$ e construir os conjuntos de vértices e arestas de $T$ separadamente.

    Para computar o conjunto de vértices, identifica-se cliques maximais em $R_k$ através da poda sucessiva das $k$-folhas de $R_k$. Esse processo pode ser visto na função {\tt pruneRk} do pacote {\tt characteristic}. Para cada vértice $v$ podado, essa função guarda uma lista $K_v \subseteq Adj(v)$ dos exatamente $k$ vértices adjacentes a $v$ que ainda não foram podados.

    Ao fim desse processo, que tem complexidade $O(nk)$, a \emph{$k$-tree} de Rényi é reduzida apenas à sua raiz $R = \{n-k+1, \cdots, n\}$.

    A partir das listas $K_i$ ($i \in V$) e da ordem em que os vértices foram podados, constrói-se o conjunto das arestas num processo de complexidade $O(nk)$ detalhado no programa 7 do artigo \cite{caminiti} cuja implementação encontra-se na função {\tt addEdges} do pacote {\tt characteristic}.

    Na nossa implementação, as arestas são representadas por duas listas (vetores), $p(v)$ e $l(v)$. Elas indicam para cada $v \in V(T)$, respectivamente, o pai de $v$ na árvore e o rótulo da aresta $(p(v), v)$.
  \end{step}

  \begin{step}
    A ideia do \emph{Dandelion Code} é enraizar a árvore $T$ no vértice $0$ e transformá-la para garantir a existência da aresta $(0, x)$. Por meio dessa transformação, o vetor de pais da árvore (transformada) vai conter duas informações inúteis (os pais de $0$ e $x$), cuja eliminação leva a uma representação da árvore com $n - 2$ rótulos.

    Escolhemos $x = \phi(\bar{q})$ onde $\bar{q} = min\{v \not \in Q\}$ e, enquanto $p(x) \neq 0$, fazemos sucessivas trocas $p(x) \leftrightarrow p(w)$, $l(x) \leftrightarrow l(w)$ escolhendo $w$ como o vértice de maior rótulo no caminho entre $0$ e $x$.

    A implementação desse processo pode ser vista na função {\tt Code} do pacote {\tt dandelion}.

    Ao final, o código $S$ é dado por uma lista ordenada de pares $(p(v), l(v)) \  \forall v \in V(T) \setminus \{0, x\}$.
  \end{step}

  \begin{step}
    Como $l_M$ foi escolhida como a folha de maior rótulo adjacente a $Q$, ela não é $\bar{q}$ (porque $\bar{q} = min\{v \not \in Q\}$ e $n \geq k + 2$). A prova formal desse fato pode ser encontrada no Lema 1 do artigo \cite{caminiti}. Além disso, $\phi(l_M)$ não estava no caminho de $0$ a $x = \phi(\bar{q})$ em $T$ (porque é uma folha).

    Como $l_M$ é adjacente a $Q$, $\phi(l_M)$ é adjacente a $0$. Portanto $(p(\phi{l_M}), l(\phi{l_M})) = (0, \varepsilon)$ pode ser removido da lista $S$ de forma que o tamanho do código passe a ser $n - k - 2$. Isso é crucial para o código ser bijetivo.

    O algoritmo retorna o par $(Q, S)$.
  \end{step}
\end{algorithm}

\subsection{Decodificação}
\label{subsec:decodificacao}

O algoritmo para decodificar um par $(Q, S) \in \mathcal{A}^n_k$ em uma \emph{$k$-tree} rotulada $T_k$ com $n$ vértices consiste numa sequência de transformações inversas às transformações usadas no algoritmo de codificação. Aqui apresentamos esse algoritmo, de complexidade $O(nk)$, indicando onde cada um dos passos pode ser encontrado na nossa implementação.

\begin{algorithm}[Algoritmo de decodificação]
  \textbf{Entrada:} um código $(Q, S)$ em $\mathcal{A}^n_k$\\
  \textbf{Saída:} uma \emph{$k$-tree} $T_k$ com $n$ vértices

  \begin{enumerate}
    \item Computar $\phi$, $\bar{q}$, $x$ e $l_M$ (definidos como no algoritmo de codificação);
    \item Inserir o par $(0, \varepsilon)$ correspondente a $l_M$ em $S$ e decodificar $S$ para obter a árvore característica $T$;
    \item Reconstruir a \emph{$k$-tree} de Rényi $R_k$ a partir de $T$;
    \item Aplicar $\phi^{-1}$ a $R_k$ para obter $T_k$.
  \end{enumerate}

  O algoritmo de decodificação é implementado pela função {\tt DecodingAlgorithm} do pacote {\tt codec}. A seguir, detalhamos os quatro passos.

  \setcounter{step}{0}

  \begin{step}
    Para computar $\phi$, $\bar{q}$, $x$ e $l_M$, os procedimentos são exatamente os mesmos usados no algoritmo de codificação.
  \end{step}

  \begin{step}
    Como já computamos $\phi$ e $l_M$ no passo anterior, inserimos o par $(0, \varepsilon)$ na posição $\phi(l_M)$ do vetor $S$.

    O procedimento para decodificar o \emph{Dandelion Code} numa árvore característica, implementado na função {\tt Decode} do pacote {\tt dandelion}, consiste em:

    \begin{enumerate}
      \item Construir o grafo a partir do código $S$, gerando vetores $p$ (de pais) e $l$ (de rótulos das arestas $(p(v), v)$);
      \item Identificar todos os ciclos do grafo e guardar num vetor $m$, para cada ciclo, o vértice com maior rótulo;
      \item Ordenar o vetor $m$ em ordem crescente e iterar nele fazendo trocas $p(x) \leftrightarrow p(m_i)$, $l(x) \leftrightarrow l(m_i)$ (para $i = 1, \cdots, |m|$).
    \end{enumerate}

    A árvore característica $T$ é dada pelo par $(p, l)$ resultante desse processo.
  \end{step}

  \begin{step}
    A reconstrução da \emph{$k$-tree} de Rényi $R_k$ a partir de $T$ foi implementada na função {\tt RenyiKtreeFrom} do pacote {\tt characteristic}.

    O processo consiste em inicializar $R_k$ com o $k$-clique $\{ n-k+1, \cdots, n \}$ e percorrer $T$ na ordem da busca em largura (a partir dos filhos do vértice de rótulo $0$) para inserir vértices em $R_k$.

    O programa 8 do artigo de Caminiti \emph{et al.} \cite{caminiti} detalha esse passo.
  \end{step}

  \begin{step}
    Para transformar a \emph{$k$-tree} de Rényi $R_k$ na \emph{$k$-tree} rotulada $T_k$, basta aplicar o inverso da permutação $\phi$. Esse processo foi implementado na função {\tt TkFrom} do pacote {\tt ktree}.
  \end{step}
\end{algorithm}

\section{Geração uniforme}
\label{sec:geracao}

Como comentamos no início deste capítulo, se temos uma codificação bijetiva que associa \emph{$k$-trees} a \emph{strings}, basta gerar \emph{strings} aleatórias para gerar \emph{$k$-trees} aleatórias.

Para gerar \emph{$k$-trees} aleatórias de forma uniforme, usamos o código de Caminiti \emph{et al.} \cite{caminiti} e o algoritmo linear para decodificar uma \emph{string} em uma \emph{$k$-tree} rotulada que apresentamos na seção \ref{sec:caminiti}.

As \emph{strings} que estamos interessados em gerar são elementos do conjunto:

$$
\mathcal{A}^n_k = { [1,n] \choose k } \times (\{ ( 0, \varepsilon ) \} \cup ([1,n-k] \times [1,k]))^{n-k-2}
$$

A função que implementamos para gerar tais \emph{strings} é {\tt randomCode}, que recebe {\tt n} e {\tt k} como parâmetros e pertence ao pacote {\tt generator}.

Primeiramente, ela sorteia $Q$ em ${ [1,n] \choose k }$ (e inicializa um \emph{Dandelion Code} vazio):

\lstinputlisting[firstline=51, lastline=59]{../../generator/generator.go}

Depois, ela gera $S$ sorteando $n-k-2$ pares em $\{ ( 0, \varepsilon ) \} \cup ([1,n-k] \times [1,k])$. Para gerar um par nesse intervalo de forma uniforme, gera-se um inteiro $r$ no intervalo $[0, (n-k)k+1)$. Se $r = 0$, então o par é $(0, \varepsilon)$. Caso contrário, o par é dado por $\left(1 + \frac{r-1}{k}, (r-1) \mod k\right)$:

\lstinputlisting[firstline=61, lastline=71]{../../generator/generator.go}

Decodificamos o código usando o algoritmo de decodificação apresentado na seção \ref{sec:caminiti} para transformar essa string $(Q, S) \in \mathcal{A}^n_k$ em uma \emph{$k$-tree} rotulada.

\section{Utilitários}

Para exemplificar como se usa a biblioteca desenvolvida nas seções anteriores, foram desenvolvidos três utilitários que se encontram no pacote {\tt examples}: {\tt code-ktree}, {\tt decode-ktree} e {\tt generate-ktree}.

Eles permitem codificar/decodificar \emph{$k$-trees} e gerar \emph{$k$-trees} aleatórias.

\subsection{code-ktree}

O utilitário {\tt code-ktree} serve para codificar \emph{$k$-trees} usando o algoritmo da subseção \ref{subsec:codificacao}. Sua entrada deve ser dada no formato\footnote{A leitura da entrada despreza espaços e quebras de linha.}:

\begin{lstlisting}
n k
m
x_1 y_1
...
x_m y_m
\end{lstlisting}

Onde:

\begin{itemize}
  \item {\tt n} é o número de vértices;
  \item {\tt k} é o parâmetro $k$ da \emph{$k$-tree};
  \item {\tt m} é o número de arestas;
  \item {\tt x\_i y\_i} corresponde à \emph{i}-ésima aresta ($0 \leq $ {\tt x\_i, y\_i} $ < n$).
\end{itemize}

Um exemplo de entrada equivalente à \emph{$k$-tree} da figura \ref{fig:rootedktree}(a) é:

\begin{lstlisting}
11 3
27
0 1 0 4 0 6 0 7
1 2 1 4 1 5 1 7 1 8 1 9 1 10
2 3 2 4 2 7 2 8 2 9 2 10
3 8 3 10 4 6
4 7
5 7 5 8
6 7
7 8
8 9 8 10
\end{lstlisting}

A saída desse utilitário é um par $(Q, S)$ no formato de entrada esperado pelo utilitário {\tt decode-ktree}, que será descrito a seguir.

\subsection{decode-ktree}

O utilitário {\tt decode-ktree} serve para decodificar um código $(Q, S)$ numa \emph{$k$-tree} usando o algoritmo da subseção \ref{subsec:decodificacao}. Sua entrada deve ser dada no formato:

\begin{lstlisting}
k
Q_1
...
Q_k
s
p_1 l_1
...
p_s l_s
\end{lstlisting}

Onde:

\begin{itemize}
  \item {\tt k} é o tamanho de $Q$;
  \item {\tt Q\_i} corresponde ao $i$-ésimo valor em $Q$;
  \item {\tt s} é o tamanho do \emph{Generalized Dandelion Code}, $|S|$;
  \item {\tt p\_i l\_i} corresponde ao $i$-ésimo valor em $S$.
\end{itemize}

Um exemplo de entrada equivalente ao código gerado pela \emph{$k$-tree} da figura \ref{fig:rootedktree}(a) é:

\begin{lstlisting}
3
1 2 8
6
0 -1
2 0
8 2
8 1
1 2
5 2
\end{lstlisting}

A saída desse utilitário é uma \emph{$k$-tree} no formato de entrada esperado pelo utilitário {\tt code-ktree}.

\subsection{generate-ktree}

O utilitário {\tt generate-ktree} serve para gerar uma \emph{$k$-tree} aleatória usando o algoritmo desenvolvido na seção \ref{sec:geracao}.

Sua entrada deve ser dada no formato:

\begin{lstlisting}
n k
\end{lstlisting}

Sua saída é uma \emph{$k$-tree} com $n$ vértices no formato de entrada esperado pelo utilitário {\tt code-ktree}.

\section{Testes, experimentos e resultados}

\subsection{Testes unitários e cobertura}

Como escrevemos na introdução deste trabalho, um dos motivos pelos quais escolhemos a linguagem \emph{Go} para a implementação foi a facilidade para escrever testes.

Todos os pacotes desenvolvidos neste trabalho possuem testes unitários que podem ser executados usando o utilitário {\tt go test}:

\begin{lstlisting}
$ go get github.com/tmadeira/tcc/...
$ go test -v github.com/tmadeira/tcc/...
=== RUN   TestTreeFrom
--- PASS: TestTreeFrom (0.00s)
=== RUN   TestRenyiKtreeFrom
--- PASS: TestRenyiKtreeFrom (0.00s)
PASS
ok  	github.com/tmadeira/tcc/characteristic	0.002s
=== RUN   TestCodingAlgorithm
--- PASS: TestCodingAlgorithm (0.00s)
=== RUN   TestDecodingAlgorithm
--- PASS: TestDecodingAlgorithm (0.00s)
PASS
ok  	github.com/tmadeira/tcc/codec	0.017s
=== RUN   TestCodeFig2C
--- PASS: TestCodeFig2C (0.00s)
=== RUN   TestDecodeFig2C
--- PASS: TestDecodeFig2C (0.00s)
=== RUN   TestDecodeFig3
--- PASS: TestDecodeFig3 (0.00s)
PASS
ok  	github.com/tmadeira/tcc/dandelion	0.002s
=== RUN   TestRandomKtree
--- PASS: TestRandomKtree (0.03s)
PASS
ok  	github.com/tmadeira/tcc/generator	0.029s
=== RUN   TestGetQ
--- PASS: TestGetQ (0.00s)
=== RUN   TestComputePhi
--- PASS: TestComputePhi (0.00s)
=== RUN   TestRelabel
--- PASS: TestRelabel (0.00s)
=== RUN   TestRkFrom
--- PASS: TestRkFrom (0.00s)
=== RUN   TestTkFrom
--- PASS: TestTkFrom (0.00s)
PASS
ok  	github.com/tmadeira/tcc/ktree	0.002s
\end{lstlisting}

Com efeito, 96\% das linhas do código são cobertas por testes, como mostra o relatório da ferramenta \emph{Coveralls}\footnote{Esse relatório pode ser visto em: \url{https://coveralls.io/github/tmadeira/tcc?branch=master}}.

\subsection{Experimentos e resultados}

\subsubsection{Corretude e uniformidade}

Para mostrar que nossa implementação gera \emph{$k$-trees} aleatórias corretamente e uniformemente, realizamos dezenas de milhares de testes com $n$ e $k$ pequenos.

Escrevemos um pequeno \emph{script} em Bash para nos auxiliar nesse experimento. Ele usa o utilitário {\tt generate-ktree} para gerar $10000$ \emph{$k$-trees} com parâmetros $(n, k)$ constantes e imprime quantas vezes cada \emph{$k$-tree} diferente foi gerada:

\begin{lstlisting}[language=bash]
i=0
while [ $i -lt 10000 ]; do
  echo $N $K | generate-ktree | xargs echo
  i=$((i+1))
done | sort | uniq -c
\end{lstlisting}

Com $n = 3$ e $k = 1$, existem $3$ \emph{$k$-trees} rotuladas distintas, como mostra a figura \ref{fig:n3k1}.

\begin{figure}
  \begin{minipage}{0.333\textwidth}
    \centering
    \scalebox{0.75}{
      \begin{tikzpicture}
          [scale=.5,auto=left,every node/.style={draw, circle, inner sep = 0pt, minimum width = 0.7cm}]
        \node (n0) at (1,1) {0};
        \node (n1) at (3,3) {1};
        \node (n2) at (1,5) {2};

        \foreach \from/\to in {n0/n1, n1/n2}
          \draw (\from) edge (\to);
      \end{tikzpicture}
    }
  \end{minipage}\begin{minipage}{0.333\textwidth}
    \centering
    \scalebox{0.75}{
      \begin{tikzpicture}
          [scale=.5,auto=left,every node/.style={draw, circle, inner sep = 0pt, minimum width = 0.7cm}]
        \node (n0) at (1,1) {0};
        \node (n1) at (3,3) {1};
        \node (n2) at (1,5) {2};

        \foreach \from/\to in {n0/n1, n0/n2}
          \draw (\from) edge (\to);
      \end{tikzpicture}
    }
  \end{minipage}\begin{minipage}{0.333\textwidth}
    \centering
    \scalebox{0.75}{
      \begin{tikzpicture}
          [scale=.5,auto=left,every node/.style={draw, circle, inner sep = 0pt, minimum width = 0.7cm}]
        \node (n0) at (1,1) {0};
        \node (n1) at (3,3) {1};
        \node (n2) at (1,5) {2};

        \foreach \from/\to in {n0/n2, n1/n2}
          \draw (\from) edge (\to);
      \end{tikzpicture}
    }
  \end{minipage}

  \caption{Representação das três \emph{$1$-trees} rotuladas distintas com $n = 3$ vértices.}
  \label{fig:n3k1}
\end{figure}

Ao executar o \emph{script} com {\tt N=3 K=1} esperamos portanto que as três \emph{$1$-trees} apareçam com uma frequência similar. O resultado que obtivemos foi:

\begin{lstlisting}
3320 3 1 2 0 1 0 2
3345 3 1 2 0 1 1 2
3335 3 1 2 0 2 1 2
\end{lstlisting}

O primeiro inteiro que aparece em cada linha é a quantidade de vezes que a \emph{$k$-tree} apareceu e o restante é a \emph{$k$-tree} gerada no formato de saída do utilitário {\tt generate-ktree} (sem quebras de linha).

Como a frequência de cada uma das $3$ \emph{$k$-trees} com $n = 3$ e $k = 1$ está similar, o experimento mostra que o algoritmo gera \emph{$k$-trees} aleatórias de forma uniforme.

Testes com outros pares $(n, k)$ também mostram frequências similares, comprovando a uniformidade. Por exemplo, existem 6 \emph{$2$-trees} com $n = 4$ vértices, como pode-se ver na figura \ref{fig:n4k2}. Rodando o \emph{script} com {\tt N=4 K=2} obtivemos:

\begin{lstlisting}
1703 4 2 5 0 1 0 2 0 3 1 2 1 3
1627 4 2 5 0 1 0 2 0 3 1 2 2 3
1573 4 2 5 0 1 0 2 0 3 1 3 2 3
1709 4 2 5 0 1 0 2 1 2 1 3 2 3
1717 4 2 5 0 1 0 3 1 2 1 3 2 3
1671 4 2 5 0 2 0 3 1 2 1 3 2 3
\end{lstlisting}

\begin{figure}
  \begin{minipage}{0.1666\textwidth}
    \centering
    \scalebox{0.75}{
      \begin{tikzpicture}
          [scale=.5,auto=left,every node/.style={draw, circle, inner sep = 0pt, minimum width = 0.7cm}]
        \node (n0) at (1,1) {0};
        \node (n1) at (1,4) {1};
        \node (n2) at (4,1) {2};
        \node (n3) at (4,4) {3};

        \foreach \from/\to in {n0/n1, n0/n2, n0/n3, n1/n2, n1/n3}
          \draw (\from) edge (\to);
      \end{tikzpicture}
    }
  \end{minipage}\begin{minipage}{0.1666\textwidth}
    \centering
    \scalebox{0.75}{
      \begin{tikzpicture}
          [scale=.5,auto=left,every node/.style={draw, circle, inner sep = 0pt, minimum width = 0.7cm}]
        \node (n0) at (1,1) {0};
        \node (n1) at (1,4) {1};
        \node (n2) at (4,1) {2};
        \node (n3) at (4,4) {3};

        \foreach \from/\to in {n0/n1, n0/n2, n0/n3, n1/n2, n2/n3}
          \draw (\from) edge (\to);
      \end{tikzpicture}
    }
  \end{minipage}\begin{minipage}{0.1666\textwidth}
    \centering
    \scalebox{0.75}{
      \begin{tikzpicture}
          [scale=.5,auto=left,every node/.style={draw, circle, inner sep = 0pt, minimum width = 0.7cm}]
        \node (n0) at (1,1) {0};
        \node (n1) at (1,4) {1};
        \node (n2) at (4,1) {2};
        \node (n3) at (4,4) {3};

        \foreach \from/\to in {n0/n1, n0/n2, n0/n3, n1/n3, n2/n3}
          \draw (\from) edge (\to);
      \end{tikzpicture}
    }
  \end{minipage}\begin{minipage}{0.1666\textwidth}
    \centering
    \scalebox{0.75}{
      \begin{tikzpicture}
          [scale=.5,auto=left,every node/.style={draw, circle, inner sep = 0pt, minimum width = 0.7cm}]
        \node (n0) at (1,1) {0};
        \node (n1) at (1,4) {1};
        \node (n2) at (4,1) {2};
        \node (n3) at (4,4) {3};

        \foreach \from/\to in {n0/n1, n0/n2, n1/n2, n1/n3, n2/n3}
          \draw (\from) edge (\to);
      \end{tikzpicture}
    }
  \end{minipage}\begin{minipage}{0.1666\textwidth}
    \centering
    \scalebox{0.75}{
      \begin{tikzpicture}
          [scale=.5,auto=left,every node/.style={draw, circle, inner sep = 0pt, minimum width = 0.7cm}]
        \node (n0) at (1,1) {0};
        \node (n1) at (1,4) {1};
        \node (n2) at (4,1) {2};
        \node (n3) at (4,4) {3};

        \foreach \from/\to in {n0/n1, n0/n3, n1/n2, n1/n3, n2/n3}
          \draw (\from) edge (\to);
      \end{tikzpicture}
    }
  \end{minipage}\begin{minipage}{0.1666\textwidth}
    \centering
    \scalebox{0.75}{
      \begin{tikzpicture}
          [scale=.5,auto=left,every node/.style={draw, circle, inner sep = 0pt, minimum width = 0.7cm}]
        \node (n0) at (1,1) {0};
        \node (n1) at (1,4) {1};
        \node (n2) at (4,1) {2};
        \node (n3) at (4,4) {3};

        \foreach \from/\to in {n0/n2, n0/n3, n1/n2, n1/n3, n2/n3}
          \draw (\from) edge (\to);
      \end{tikzpicture}
    }
  \end{minipage}

  \caption{Representação das seis \emph{$2$-trees} rotuladas distintas com $n = 4$ vértices.}
  \label{fig:n4k2}
\end{figure}

E com {\tt N=5 K=3} obtivemos:

\begin{lstlisting}
  970 5 3 9 0 1 0 2 0 3 0 4 1 2 1 3 1 4 2 3 2 4
 1023 5 3 9 0 1 0 2 0 3 0 4 1 2 1 3 1 4 2 3 3 4
 1009 5 3 9 0 1 0 2 0 3 0 4 1 2 1 3 1 4 2 4 3 4
 1014 5 3 9 0 1 0 2 0 3 0 4 1 2 1 3 2 3 2 4 3 4
  994 5 3 9 0 1 0 2 0 3 0 4 1 2 1 4 2 3 2 4 3 4
 1019 5 3 9 0 1 0 2 0 3 0 4 1 3 1 4 2 3 2 4 3 4
 1008 5 3 9 0 1 0 2 0 3 1 2 1 3 1 4 2 3 2 4 3 4
 1000 5 3 9 0 1 0 2 0 4 1 2 1 3 1 4 2 3 2 4 3 4
  978 5 3 9 0 1 0 3 0 4 1 2 1 3 1 4 2 3 2 4 3 4
  985 5 3 9 0 2 0 3 0 4 1 2 1 3 1 4 2 3 2 4 3 4
\end{lstlisting}

% TODO: colocar figura das k-trees com N=5, K=3

%\subsubsection{Tempo linear}
%
%Para checar se nossa implementação gera \emph{$k$-trees} em tempo $O(nk)$ como esperado, geramos \emph{$k$-trees} com diferentes valores de $n$ e de $k$ várias vezes, calculamos a média e o desvio padrão dos tempos de execução e comparamos os resultados com outros pares de parâmetros $(n, k)$.
%
%\vspace{2em}
%
%A testar/escrever. % TODO


  \cleardoublepage

  \chapter{Aprendizado de redes bayesianas}
\label{cap:aprendizado}

Neste capítulo apresentamos o problema de aprender redes bayesianas e descrevemos um método desenvolvido por Nie \emph{et al.} \cite{maua} para resolvê-lo que é baseado na amostragem de \emph{$k$-trees}. Mostramos como a geração uniforme de \emph{$k$-trees} desenvolvida no capítulo \ref{cap:geracao} foi utilizada nesse processo e comparamos os resultados obtidos com os resultados de outros trabalhos.

\section{Motivação}

Redes bayesianas são modelos probabilísticos gráficos que representam distribuições de probabilidade conjunta e são usados para raciocinar em situações com incerteza.

Formalmente \cite{maua}, seja $N = \{ 1, \cdots, n \}$ e seja $X = \{X_i : i \in N\}$ um conjunto de variáveis aleatórias $X_i$ tomando valores em conjuntos finitos $\mathcal{X}_i$. Uma \textbf{rede bayesiana} é uma tripla $(X, G, \theta)$, onde $G = (V, E)$ é um DAG (que chamamos de \textbf{estrutura} da rede bayesiana) cujos vértices correspondem a variáveis em $X$ e $\theta = \{\theta_i(x_i, x_{\pi_i})\}$ é um conjunto de parâmetros numéricos especificando valores de probabilidade condicional $\theta_i(x_i, x_{\pi_i}) = P(x_i | x_{\pi_i})$ para todo vértice $i \in V$, valor $x_i \in X_i$ e atribuição $x_{\pi_i}$ para os pais $\pi_i$ de $X_i$ (em $G$).

Um exemplo de rede bayesiana é mostrado na figura \ref{fig:bayes}. As arestas codificam nossa intuição sobre a forma como o mundo funciona: as performances do aluno e do supervisor são determinadas independentemente; a nota do TCC (trabalho de conclusão de curso) depende desses dois fatores; a média ponderada do aluno depende apenas da sua performance no curso; a carta de recomendação escrita pelo supervisor sobre o aluno depende da nota do TCC.

\begin{figure}
  \centering
  \begin{tikzpicture}
      [scale=.5,auto=left,every node/.style={draw, ellipse, inner sep = 3pt, minimum width = 0.72cm, align = center}]

      \node[fill=gray!30] (or) at (1,9) {Supervisor};
      \node[fill=gray!30] (al) at (7,9) {Aluno};
      \node[fill=gray!30] (no) at (2,5) {Nota TCC};
      \node[fill=gray!30] (me) at (10,5) {Média ponderada};
      \node[fill=gray!30] (ca) at (0,1) {Carta de recomendação};

      \path (or) edge[-triangle 60] (no)
        (al) edge[-triangle 60] (no)
        (al) edge[-triangle 60] (me)
        (no) edge[-triangle 60] (ca);

      \node[draw=none,inner sep = 0pt,above=of or]
      {
        \begin{tabular}{|c|c|} \hline
          $o_0$ & $o_1$ \\ \hline
          0,5 & 0,5 \\ \hline
        \end{tabular}
      };

      \node[draw=none,inner sep = 0pt,above=of al]
      {
        \begin{tabular}{|c|c|} \hline
          $a_0$ & $a_1$ \\ \hline
          0,5 & 0,5 \\ \hline
        \end{tabular}
      };

      \node[draw=none,inner sep = 0pt,below=of me]
      {
        \begin{tabular}{|c|c|c|} \hline
                & $m_0$ & $m_1$ \\ \hline
          $a_0$ & 0,8 & 0,2 \\ \hline
          $a_1$ & 0,2 & 0,8 \\ \hline
        \end{tabular}
      };

      \node[draw=none,inner sep = 0pt,left=of no]
      {
        \begin{tabular}{|c|c|c|} \hline
                & $n_0$ & $n_1$ \\ \hline
          $a_0,o_0$ & 0,9 & 0,1 \\ \hline
          $a_0,o_1$ & 0,7 & 0,3 \\ \hline
          $a_1,o_0$ & 0,5 & 0,5 \\ \hline
          $a_1,o_1$ & 0,1 & 0,9 \\ \hline
        \end{tabular}
      };

      \node[draw=none,inner sep = 0pt,below=of ca]
      {
        \begin{tabular}{|c|c|c|} \hline
                & $c_0$ & $c_1$ \\ \hline
          $n_0$ & 0,9 & 0,1 \\ \hline
          $n_1$ & 0,1 & 0,9 \\ \hline
        \end{tabular}
      };
  \end{tikzpicture}

  \caption{Exemplo de rede bayesiana com distribuições de probabilidade condicional.}
  \label{fig:bayes}
\end{figure}

Redes bayesianas são geralmente usadas para fazer inferências como computar a probabilidade de alguma variável depois que alguma evidência é observada. Por exemplo, uma rede bayesiana pode representar as relações de probabilidade entre doenças e sintomas. Observados alguns sintomas, a rede pode ser usada para computar a probabilidade da presença das doenças.

Para dar um exemplo mais concreto, com a rede bayesiana mostrada na figura \ref{fig:bayes} pode-se modificar a crença sobre a performance do aluno ou do supervisor com base na nota do TCC.

\vspace{2em}

Aprender uma rede bayesiana se refere ao processo de inferir a estrutura (ou seja, o DAG) dela a partir de dados. Como mostra Chickering \cite{chickering}, este é um problema NP-completo.

O aprendizado de redes bayesianas costuma servir para realizar inferências em situações com incerteza. O artigo de Nie \emph{et al.} \cite{maua} mostra que tais inferências são NP-difíceis até mesmo aproximadamente e todos os algoritmos conhecidos (exatos e comprovadamente bons) têm uma complexidade de pior caso exponencial no \emph{treewidth}.

Além disso, resultados empíricos sugerem que limitar o \emph{treewidth} pode melhorar a performance dos modelos e há evidências de que limitar a \emph{treewidth} da estrutura de uma rede bayesiana não causa perdas significativas na expressividade do modelo para conjuntos de dados reais (também visto em \cite{maua}).

Por isso, estamos interessados em fixar $k$ e aprender redes bayesianas cuja estrutura tem \emph{treewidth} limitado a $k$.

\vspace{2em}

A fim de identificar o ``melhor'' DAG para um determinado conjunto de dados, vamos supôr que há uma função de \emph{score} $s(G)$ que atribui uma pontuação para cada DAG $G$ em tempo constante. Segundo \cite{nie}, as funções de \emph{score} costumam poder ser escritas como a soma de funções de \emph{score} locais, ou seja,

$$s(G) = \sum_{i \in N} s_i(X_{\pi_i}).$$

Para cada variável, sua pontuação só depende do seu conjunto de pais. Ou seja, nosso problema é encontrar $G^*$ tal que

$$G^* = \argmax_{G \in \mathcal{G}_{n,k}} \sum_{i \in N} s_i(\pi_i),$$

onde $\mathcal{G}_{n,k}$ é o conjunto de todos os DAGs de \emph{treewidth} não maiores que $k$.

Mesmo esse problema é NP-difícil, como mostram Korhonen e Parviainen \cite{korhonen}. Entretanto, o artigo \cite{maua} mostra um método aproximado para aprender redes bayesianas com \emph{treewidth} limitado que é baseado em amostrar \emph{$k$-trees} e encontrar DAGs cujo grafo moral é um subgrafo dessas \emph{$k$-trees}.

Tal método funciona com domínios grandes e \emph{treewidth} alto. No artigo é mostrado empiricamente que ele tem um desempenho muito bom numa coleção de conjuntos de dados públicos.

\section{Aprendizado por amostragem de \emph{$k$-trees}}
\label{sec:aprendizado}

A ideia para aprender um DAG por meio da amostragem de \emph{$k$-trees} baseia-se em, para cada \emph{$k$-tree} $T_k$ amostrada, construir uma ordem parcial $\sigma$ dos vértices e fazer com que o DAG $G$ seja consistente com ela e com $T_k$.

Como explica \cite{maua}, \emph{``$\sigma$ restringe as ordenações topológicas válidas para os vértices de $G$''} de forma que garante que ele não tenha ciclos. Por outro lado, fazer com que $G$ seja subgrafo de $T_k$ garante que seu \emph{treewidth} não exceda $k$.

Construímos a ordem parcial $\sigma$ a partir do enraizamento da \emph{$k$-tree} num $k$-clique qualquer. Primeiro, sorteamos uniformemente a ordem dos vértices do $k$-clique raiz. A partir dele, visitamos os vértices da árvore na ordem da busca em largura. Cada vértice que visitamos é inserido num lugar aleatório da ordem, que é definido de forma uniforme. Determinado esse lugar, selecionamos os melhores pais para ele dentre os vértices adjacentes predecessores já visitados e atualizamos os melhores pais dos vértices adjacentes.

O passo 3 do algoritmo de decodificação de \emph{Dandelion Code} em \emph{$k$-tree} apresentado na subseção \ref{subsec:decodificacao} percorre a árvore exatamente como gostaríamos para construir uma \emph{$k$-tree} de Rényi a partir de uma árvore característica, o que sugere que construamos o DAG diretamente a partir da árvore característica.

O algoritmo de aprendizado fica como segue:

\begin{algorithm}[Algoritmo para aprender estrutura]
  \textbf{Entrada:} número de variáveis $n$, a \emph{treewidth} desejada $k$ e uma função de \emph{score} $s_i$ para cada $i \in [0, n)$\\
  \textbf{Saída:} um DAG $G^{\text{melhor}}$

  \begin{enumerate}
    \item Inicializar $G^{\text{melhor}}$ como um grafo vazio com $s(G^{\text{melhor}}) = -\infty$.
    \item Repetir até atingir um determinado número de iterações:
      \begin{enumerate}
        \item Gerar $(Q, S) \in \mathcal{A}^n_k$ conforme seção \ref{sec:geracao};
        \item Decodificar $(Q, S)$ na árvore característica $T$ usando o algoritmo da subseção \ref{subsec:decodificacao};
        \item Determinar uniformemente uma ordem pros vértices do $k$-clique raiz (ou seja, os vértices correspondentes a $\phi^{-1}(v)$ para os vértices $v \in \{n-k+1, n-k+2, \cdots, n\}$ correspondentes a raiz da \emph{$k$-tree} de Rényi) e usar a função de \emph{score} para calcular os melhores pais para cada um deles entre os vértices que os precedem;
        \item Percorrer $T$ como no passo 3 do algoritmo \ref{subsec:decodificacao} a partir dos vértices ligados ao $k$-clique raiz: para cada $v$, é sorteado um lugar para ele em $\sigma$ e selecionado seu melhor conjunto de pais dentre os vértices predecessores (em $\sigma$) adjacentes já visitados, assim como são atualizados os melhores pais dos vértices sucessores (em $\sigma$) adjacentes já visitados;
        \item Se $\left(\sum_{i \in [0,n)} s_i(\pi^G_{i})\right) = s(G) > s(G^{\text{melhor}})$, atualiza $G^{\text{melhor}} = G$.
      \end{enumerate}
  \end{enumerate}
\end{algorithm}

\section{Experimentos e resultados}

O algoritmo descrito na seção \ref{sec:aprendizado} foi implementado por João de Santana Brito Junior\footnote{Aluno de mestrado orientado pelo Prof. Dr. Denis Deratani Mauá. E-mail: \href{mailto:joaojr@ime.usp.br}{joaojr@ime.usp.br}} e seu código, que faz uso da biblioteca para gerar \emph{$k$-trees} que foi desenvolvida no capítulo \ref{cap:geracao}, está disponível em \url{https://github.com/britojr/bn}.

Usamos essa implementação para testar o aprendizado por amostragem uniforme de \emph{$k$-trees} com cinco conjuntos de dados reais contendo de 64 a 1556 variáveis. Rodamos os experimentos num computador com processador Intel Core i5 2.6GHz e 8GB RAM.

A tabela \ref{tab:tempos} mostra quanto tempo levou o aprendizado com diferentes parâmetros $k$ e número de interações.

\begin{table}
  \centering
  % TODO
  %\begin{tabular}
  %\end{tabular}

  \caption{Descrição da tabela}
  \label{tab:tempos}
\end{table}

A tabela \ref{tab:comparacao} mostra os \emph{scores} normalizados obtidos nos nossos experimentos e os compara com os resultados do artigo de Perez e Mauá \cite{perez}, que usa uma abordagem de aprendizado que tem resultados comparáveis ao estado da arte para domínios grandes.

\begin{table}
  \centering
  % TODO
  %\begin{tabular}
  %\end{tabular}

  \caption{Descrição da tabela}
  \label{tab:comparacao}
\end{table}

  \cleardoublepage

  \chapter{Conclusão}
\label{cap:conclusao}

O principal produto deste trabalho de conclusão de curso é a biblioteca implementada em \emph{Go} para codificação e geração uniforme de \emph{$k$-trees} em tempo linear desenvolvida no capítulo \ref{cap:geracao}. Os experimentos realizados no capítulo \ref{cap:aprendizado} mostram como essa biblioteca pode ser usada na prática no aprendizado da estrutura de redes bayesianas com \emph{treewidth} limitado a partir da amostragem de \emph{$k$-trees}. O método usado é eficiente e funciona com grandes domínios e limite alto de \emph{treewidth}.

Aprendizagem computacional é uma área muito contemporânea e fértil, cuja aplicação é cada vez mais difundida. Muitos dos conceitos estudados aqui são recentes e ainda pouco explorados. De fato, os dois principais artigos usados para embasar a codificação de \emph{$k$-trees} (Caminiti \emph{et al.}, \cite{caminiti}) e o aprendizado de redes bayesianas por meio da amostragem de \emph{$k$-trees} (Nie \emph{et al.}, \cite{maua}) foram publicados em 2010 e 2014, respectivamente. O segundo ressalta, na sua conclusão, que simultaneamente ao seu desenvolvimento foram publicados independentemente outros trabalhos intimamente relacionados.

Uma extensão interessante deste trabalho seria estudar a geração de DAGs de \emph{treewidth} limitado de maneira uniforme. A geração uniforme de \emph{$k$-trees} não resolve esse problema, porque mais de um DAG pode ser gerado a partir da mesma \emph{$k$-tree} e um DAG é a \emph{partial $k$-tree} de mais de uma \emph{$k$-tree}. Até onde sabemos, não há trabalhos publicados nessa direção.

Vale ainda ressaltar que se por um lado o artigo \cite{maua} diz que \emph{``amostragem uniforme é uma propriedade desejada, já que garante uma boa cobertura do espaço e é superior a outras opções se não há informação prévia sobre o espaço de busca''}, há outras amostragens que podem ser testadas e comparadas em trabalhos futuros.

Com efeito, o artigo \cite{nie}, de 2015, diz que \emph{``amostragem uniforme gera cada amostra independentemente e ignora totalmente amostras anteriores, o que faz com que seja possível que ela gere exatamente a mesma amostra duas vezes, ou ao menos amostras que são muito parecidas uma com a outra''}. Ele define e sugere que se use uma \emph{Distance Preferable Sampling} que descarte amostras que sejam muito parecidas às geradas anteriormente durante o processo de geração.

Por fim, a linguagem escolhida para fazer as implementações, \emph{Go}, se mostrou bastante satisfatória. Embora nova, ela é bastante madura e acreditamos que suas convenções (destacadas no capítulo \ref{cap:introducao}) facilitem o reaproveitamento futuro do código desenvolvido.

  \cleardoublepage

\backmatter

  \bibliographystyle{plain}
  \bibliography{referencias}

\end{document}
