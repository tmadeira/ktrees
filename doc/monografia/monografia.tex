\documentclass[a4paper,12pt]{book}

\usepackage[utf8]{inputenc}
\usepackage[brazil]{babel}
\usepackage{emptypage}
\usepackage{hyperref}
\usepackage{natbib}
\usepackage{amsmath,amssymb,amsfonts,amsthm}
\usepackage{tikz}
\usepackage{setspace}

\newtheoremstyle{definition}
{\topsep}{\topsep}
{}{}
{\bfseries}{}
{ }
{\thmname{#1}~\thmnumber{#2}\thmnote{ (#3)}.}

\theoremstyle{definition}
\newtheorem{definition}{Definição}

\newtheoremstyle{algorithm}
{\topsep}{\topsep}
{}{}
{\scshape}{}
{ }
{\thmnote{#3}\\}

\theoremstyle{algorithm}
\newtheorem{algorithm}{Algoritmo}

\newtheoremstyle{step}
{\topsep}{\topsep}
{}{}
{\itshape}{}
{1em}
{\thmname{#1}~\thmnumber{#2}.}

\theoremstyle{step}
\newtheorem{step}{Passo}

\frenchspacing
\linespread{1.5}

\title{Geração uniforme de \emph{k-trees} para aprendizado de redes bayesianas}
\author{Tiago Madeira}

\begin{document}

\frontmatter

  \thispagestyle{empty}
  \begin{center}
  \vspace*{1cm}
  Universidade de São Paulo\\
  Instituto de Matemática e Estatística\\
  Bachalerado em Ciência da Computação

  \vspace*{3cm}
  {\Large Tiago Madeira}

  \vspace{3cm}
  {
    \Large \bfseries
    Geração uniforme de \emph{k-trees} para aprendizado de redes bayesianas
  }

  \vspace{3cm}
  Supervisor: Prof. Dr. Denis Deratani Mauá

  \vspace{2cm}
  São Paulo

  Novembro de 2016
\end{center}

  \cleardoublepage\thispagestyle{empty}

  \pagenumbering{roman}
  \chapter*{Resumo}

Este trabalho de conclusão de curso consiste num estudo sobre amostragem uniforme de \emph{$k$-trees} e seu uso no aprendizado da estrutura de redes bayesianas com \emph{treewidth} limitado. Foi implementado um algoritmo para codificar e decodificar \emph{$k$-trees} de forma bijetiva em tempo linear. O trabalho mostra como aprender grafos acíclicos dirigidos cujo grafo moral é um subgrafo das \emph{$k$-trees} geradas. Experimentos comparam esse método para aprender estruturas com o estado da arte.

\vspace{1em}

\noindent \textbf{Palavras-chave:} \emph{$k$-trees}, codificação de grafos, amostragem uniforme, redes bayesianas, aprendizado de estrutura, \emph{treewidth} limitado


  \chapter*{Abstract}

This work is a study about uniform sampling of $k$-trees and its use to learn Bayesian networks with bounded treewidth. We have implemented algorithms for bijective linear time coding and decoding of $k$-trees. This work shows how to learn directed acyclic graphs whose moral graph is a subgraph of the generated $k$-trees. Experiments compare this method of structure learning with the state of the art.

\vspace{1em}

\noindent \textbf{Keywords:} $k$-trees, graph coding, uniform sampling, Bayesian networks, structure learning, bounded treewidth


  \tableofcontents
  \cleardoublepage

\mainmatter

  \chapter{Introdução}
\label{cap:introducao}

Em teoria dos grafos, \emph{$k$-trees} são consideradas uma generalização de árvores. Há interesse considerável em desenvolver ferramentas eficientes para manipular essa classe de grafos, porque todo grafo com \emph{treewidth} $k$ é um subgrafo de uma \emph{$k$-tree} e muitos problemas NP-completos podem ser resolvidos em tempo polinomial quando restritos a grafos com \emph{treewidth} limitada.

Com efeito, o artigo de Arnborg e Proskurowski \cite{arnborg} apresenta algoritmos para resolver em tempo linear problemas como, dado um grafo com \emph{treewidth} limitada:

\begin{itemize}
  \item Encontrar o tamanho máximo dos seus conjuntos independentes;
  \item Computar o tamanho mínimo dos seus conjuntos dominantes;
  \item Calcular seu número cromático; e
  \item Determinar se ele tem um ciclo hamiltoniano.
\end{itemize}

O problema que desperta nosso interesse em \emph{$k$-trees} é a inferência em redes bayesianas.

Uma rede bayesiana é um modelo probabilístico em grafo usado para raciocinar e tomar decisões em situações com incerteza através de técnicas de inteligência artificial e aprendizagem computacional. Ela representa uma distribuição de probabilidade multivariada num DAG (grafo acíclico dirigido) no qual os vértices correspondem às variáveis aleatórias do domínio e as arestas correspondem, intuitivamente, a influência de uma variável sobre outra.

Segundo Koller e Friedman \cite{koller}, a inferência em redes bayesianas em geral é NP-difícil; porém, se seu DAG possui \emph{treewidth} limitado, a inferência pode ser realizada em tempo polinomial. Daí a importância de aprender redes bayesianas que tenham \emph{treewidth} limitada.

A partir dessa motivação, este trabalho de conclusão de curso consistiu em estudar os conceitos de teoria dos grafos relacionados a \emph{$k$-trees} e implementar um algoritmo para gerar \emph{$k$-trees} de forma uniforme que possam ser usadas no aprendizado de redes bayesianas.

\section{Código desenvolvido}

As implementações deste trabalho foram realizadas na linguagem \emph{Go}\footnote{\emph{The Go Programming Language:} \url{https://golang.org/}}. \emph{Go} é uma linguagem de código aberto criada em 2007. Ela é compilada e usa tipagem estática como o C, mas por ser uma linguagem muito nova tem \emph{garbage collection} e recursos para programação concorrente.

Escolhemos \emph{Go} porque ela tem boa performance e é agradável de usar. Tem sistemas de pacotes ({\tt go get}), testes ({\tt go test}) e documentação (\emph{GoDoc}\footnote{\emph{GoDoc:} \url{https://godoc.org/}}) padronizados facilitando que os códigos sejam testados e reutilizados. Produz código limpo e padronizado (identação, espaçamento e outros detalhes de estilo são automatizados pela ferramenta {\tt gofmt} que vem com ela).

Todo o código desenvolvido neste trabalho está num repositório público no \emph{GitHub}\footnote{\emph{GitHub:} \url{https://github.com/}} cujo endereço é \url{https://github.com/tmadeira/tcc/}.

A documentação de todas as estruturas e funções declaradas no código está disponível em \url{https://godoc.org/github.com/tmadeira/tcc}.

Para baixar o código, rodar os testes e instalar os utilitários, recomenda-se usar as ferramentas da linguagem \emph{Go}:

\begin{lstlisting}
$ export ${GOPATH:=$HOME/go}
$ mkdir -p $GOPATH
$ go get github.com/tmadeira/tcc/...
$ go test -v github.com/tmadeira/tcc/...
$ go install github.com/tmadeira/tcc/examples/...
\end{lstlisting}

\section{Organização da monografia}

No capítulo \ref{cap:fundamentos}, apresentamos definições fundamentais de teoria dos grafos, teoria da probabilidade e redes bayesianas que o leitor deve conhecer para compreender o trabalho.

No capítulo \ref{cap:geracao}, apresentamos o problema de codificar \emph{$k$-trees}, discutimos os algoritmos lineares para codificar e decodificar \emph{$k$-trees} propostos no artigo \emph{``Bijective Linear Time Coding and Decoding for $k$-Trees''} \cite{caminiti}, explicamos como eles foram implementados neste trabalho para gerar \emph{$k$-trees} aleatórias uniformemente e apresentamos o resultado que obtivemos através de experimentos.

No capítulo \ref{cap:aprendizado}, explicamos como as \emph{$k$-trees} que geramos no capítulo \ref{cap:geracao} foram usadas para aprender redes bayesianas a partir do arcabouço desenvolvido no artigo \emph{``Advances in Learning Bayesian Networks of Bounded Treewidth''} \cite{maua}.

No capítulo \ref{cap:conclusao}, comparamos os resultados obtidos com o estado da arte e listamos as conclusões do trabalho.

  \cleardoublepage

  \chapter{Fundamentos}
\label{cap:fundamentos}

Neste capítulo, apresentamos algumas definições que o leitor deve conhecer para compreender o trabalho.

Partimos do pressuposto de que o leitor conhece notações básicas de conjuntos.

\begin{definition}[grafo]
  \cite{defgrafo}
  Um grafo é um par ordenado $G = (V, E)$. Os elementos de $V$ são chamados de vértices de $G$. Os elementos de $E$ são chamados de arestas de $G$ e consistem em pares (não-ordenados) de vértices. Dados $u, v \in V$, se $(u, v) \in E$ dizemos que $u$ e $v$ são adjacentes em $G$.
\end{definition}

\begin{definition}[grafo completo]
  \cite{defgrafocompleto}
  Um grafo $G = (V, E)$ é dito completo se $(u, v) \in E$ para todo $u, v \in V, u \neq v$.
\end{definition}

\begin{definition}[subgrafo induzido]
  \cite{defsubgrafo}
  Dado um grafo $G = (V, E)$ e um subconjunto $V'$ de $V$, o subgrafo de $G$ induzido por $V'$, $G' = (V', E')$, é o grafo formado pelos vértices $V' \subseteq V$ e arestas que só contém elementos de $V'$, ou seja, $E' = \{(u, v) \in E | u, v \in V'\}$.
\end{definition}

\begin{definition}[caminho]
  \cite{defcaminho}
  Dado um grafo $G = (V, E)$, um caminho é uma sequência de arestas que conectam uma sequência de vértices adjacentes distintos.
\end{definition}

\begin{definition}[distância]
  \cite{defdistancia}
  Dado um grafo $G = (V, E)$ e dois vértices $(u, v) \in V$, a distância entre $u$ e $v$ é o número de arestas num menor caminho que os conecte.
\end{definition}

\begin{definition}[árvore]
  \cite{defarvore}
  Dado um grafo $G = (V, E)$, dizemos que ele é uma árvore se cada dois vértices $u, v \in V$ são conectados por exatamente um caminho.
\end{definition}

\begin{definition}[$k$-clique]
  \label{def:kclique}
  \cite{defkclique} Seja $G = (V, E)$ um grafo. Um $k$-clique é um subconjunto dos vértices, $C \subseteq V$, tal que $(u, v) \in E \ \forall \ u, v \in C, u \neq v$ (ou seja, tal que o subgrafo induzido por $C$ é completo).
\end{definition}

\begin{definition}[$k$-tree e $k$-tree enraizada]
  \label{def:ktree}
  \cite{harary} Uma $k$-tree é definida da seguinte forma recursiva:

  \begin{enumerate}
    \item Um grafo induzido por um $k$-clique é uma $k$-tree.
    \item Se $T_k' = (V, E)$ é uma $k$-tree, $K \subseteq V$ é um $k$-clique e $v \not \in V$, então $T_k = (V \cup \{v\}, E \cup \{(v,x) \ | \  x \in K\})$ é uma $k$-tree.
  \end{enumerate}

  Uma $k$-tree enraizada é uma $k$-tree com um $k$-clique destacado $R = \{r_1, r_2, \cdots, r_k\}$ que é chamado de \emph{raiz} da $k$-tree enraizada.
\end{definition}

\begin{definition}[$k$-tree de Rényi]
  \cite{renyi} Uma $k$-tree de Rényi $R_k$ é uma $k$-tree enraizada com $n$ vértices rotulados em $[1, n]$ e raiz $R = \{n-k+1, n-k+2, \cdots, n\}$.
\end{definition}

\begin{definition}[esqueleto de uma $k$-tree enraizada]
  \label{def:skeleton}
  \cite{caminiti} O esqueleto de uma $k$-tree enraizada $T_k$ com raiz $R$, denotado por $S(T_k, R)$, é definido da seguinte forma recursiva:

  \begin{enumerate}
    \item Se $T_k$ é apenas o $k$-clique $R$, seu esqueleto é uma árvore com um único vértice $R$.
    \item Dada uma $k$-tree enraizada $T_k$ com raiz $R$, obtida por $T_k'$ enraizada em $R$ através da adição de um novo vértice $v$ conectado a um $k$-clique $K$ (ver definição \ref{def:ktree}), seu esqueleto $S(T_k, R)$ é obtido adicionando a $S(T_k', R)$ um novo vértice $X = \{v\} \cup K$ e uma nova aresta $(X, Y)$, onde $Y$ é o vértice de $S(T_k', R)$ que contém $K$ com uma distância mínima da raiz. Chamamos $Y$ de pai de $X$.
  \end{enumerate}
\end{definition}

\begin{definition}[árvore característica]
  \cite{caminiti} A árvore característica $T(T_k, R)$ de uma $k$-tree enraizada $T_k$ com raiz $R$ é obtida rotulando os vértices e arestas de $S(T_k, R)$ da seguinte forma:

  \begin{enumerate}
    \item O vértice $R$ é rotulado $0$ e cada vértice $\{v\} \cup K$ é rotulado $v$;
    \item Cada aresta do vértice $\{v\} \cup K$ ao seu pai $\{v'\} \cup K'$ é rotulada com o índice do vértice em $K'$ (visualizando-o como um conjunto ordenado) que não aparece em $K$. Quando o pai é $R$ a aresta é rotulada $\epsilon$.
  \end{enumerate}

  Note que a existência de um único vértice em $K' \setminus K'$ é garantida pela definição \ref{def:skeleton}. De fato, $v'$ precisa aparecer em $K$, caso contrário $K' = K$ e o pai de $\{v'\} \cup K'$ contém $K$. Isso contradiz o fato de que cada vértice em $S(T_k, R)$ é ligado à distância mínima da raiz.
\end{definition}


  \cleardoublepage

  \chapter{Geração aleatória de \emph{$k$-trees}}
\label{cap:geracao}

O problema de gerar \emph{$k$-trees} está intimamente relacionado ao problema de codificá-las e decodificá-las. De fato, se há uma codificação bijetiva que associa \emph{$k$-trees} a \emph{strings}, basta gerar \emph{strings} aleatórias para gerar \emph{$k$-trees} aleatórias.

Neste capítulo, apresentamos o problema de codificar \emph{$k$-trees}, discutimos a solução linear para codificar e decodificar \emph{$k$-trees} de forma bijetiva proposta por Caminiti et al. \cite{caminiti}, explicamos como ela foi implementada neste trabalho para gerar \emph{$k$-trees} aleatórias e mostramos os resultados obtidos.

\section{Codificando árvores e \emph{$k$-trees}}

O problema de codificar árvores já foi amplamente estudado na literatura. Como destaca Caminiti et al. \cite{caminiti}:

\begin{quotation}
  Codificar árvores rotuladas por meio de \emph{strings} de rótulos de vértices é uma alternativa interessante à representação usual de estruturas de dados de árvore na memória e tem muitas aplicações práticas (por exemplo, algoritmos evolucionários sobre árvores, geração aleatória de árvores, compressão de dados e computação do volume de floresta de grafos). Diversos códigos bijetivos diferentes que realizam associações entre árvores rotuladas e \emph{strings} de rótulos foram introduzidas. De um ponto de vista algorítmico, o problema foi cuidadosamente investigado e algoritmos ótimos de codificação e decodificação desses códigos são conhecidos.
\end{quotation}

Em 1889, Cayley \cite{cayley} demonstrou que para um conjunto de $n$ vértices distintos existem $n^{n-2}$ árvores possíveis. Desde lá, foram criados vários códigos para associar \emph{strings} e árvores.

Um dos mais conhecidos é o código de Prüfer \cite{prufer}, que surgiu em 1918 e é bijetivo, associando cada árvore (rotulada) de $n$ vértices a uma lista distinta de comprimento $n-2$ no alfabeto dos rótulos da árvore.

Codificar uma árvore usando o código de Prüfer é trivial: basta remover iterativamente as folhas da árvore até que apenas dois vértices sobrem, escolhendo sempre a folha de memor rótulo. Quando uma folha é removida, adiciona-se ao código o rótulo do seu vizinho.

A figura \ref{fig:prufer} exemplifica a codificação de Prüfer mostrando uma árvore cujo o código resultante do algoritmo é $\{4, 4, 4, 5\}$.

\begin{figure}
  \centering
  \begin{tikzpicture}
      [scale=.6,auto=left,every node/.style={draw, circle, inner sep = 0pt, minimum width = 0.72cm}]
    \node (n1) at (1,8) {1};
    \node (n2) at (4,8) {2};
    \node (n3) at (6,6) {3};
    \node (n4) at (3,5) {4};
    \node (n5) at (3.5,3) {5};
    \node (n6) at (0,2) {6};

    \foreach \from/\to in {n1/n4, n2/n4, n3/n4, n4/n5, n5/n6}
      \draw (\from) edge (\to);
  \end{tikzpicture}

  \caption{A árvore rotulada equivalente ao código de Prüfer $\{4, 4, 4, 5\}$.}
  \label{fig:prufer}
\end{figure}

\vspace{2em}

\emph{$k$-trees} \cite{harary} são consideradas uma generalização de árvores. Há interesse considerável em desenvolver ferramentas eficientes para manipular essa classe de grafos, porque todo grafo com \emph{treewidth} $k$ é um subgrafo de uma \emph{$k$-tree} e muitos problemas NP-completos podem ser resolvidos em tempo polinomial quando restritos a grafos com \emph{treewidth} limitada, como destacado na \textbf{Introdução} deste trabalho.

Há estudos sobre a codificação de \emph{$k$-trees} há pelo menos quatro décadas. Em 1970, Rényi e Renýi apresentaram uma codificação redundante (ou seja, não bijetiva) para um subconjunto de \emph{$k$-trees} rotuladas que chamamos de \emph{$k$-trees} de Rényi e que são definidas como segue:

\begin{definition}[\emph{$k$-tree} de Rényi]
  \cite{renyi} Uma \emph{$k$-tree} de Rényi $R_k$ é uma \emph{$k$-tree} enraizada com $n$ vértices rotulados em $[1, n]$ e raiz $R = \{n-k+1, n-k+2, \cdots, n\}$.
\end{definition}

Entretanto, até onde sabemos, apenas em 2008 surgiu um código bijetivo para \emph{$k$-trees} com algoritmos lineares de codificação e decodificação. Foram esses algoritmos, propostos por Caminiti et al. \cite{caminiti}, que implementamos neste trabalho.

\section{A solução de Caminiti et al.}

O artigo \emph{``Bijective Linear Time Coding and Decoding for $k$-Trees''} \cite{caminiti} apresenta um código bijetivo para \emph{$k$-trees} rotuladas, juntamente a algoritmos lineares para realizar a codificação e a decodificação.

O código é formado por uma permutação de tamanho $k$ e uma generalização do \emph{Dandelion Code} \cite{egecioglu}, que consiste em $n-k-2$ pares (onde $n$ é o número de vértices) definidos no conjunto $\{ ( 0, \varepsilon ) \} \cup ([1,n-k] \times [1,k])$. Portanto, dizemos que a codificação das \emph{$k$-trees} associa elementos em $\mathcal{T}^n_k$ (conjunto das \emph{$k$-trees} com $n$ vértices) com elementos em:

$$
\mathcal{A}^n_k = { [1,n] \choose k } \times (\{ ( 0, \varepsilon ) \} \cup ([1,n-k] \times [1,k]))^{n-k-2}
$$

Os algoritmos consistem em uma série de transformações. Para compreendê-los, é necessário definir esqueleto de uma \emph{$k$-tree} enraizada e árvore característica:

\begin{definition}[esqueleto de uma \emph{$k$-tree} enraizada]
  \label{def:skeleton}
  \cite{caminiti} O esqueleto de uma \emph{$k$-tree} enraizada $T_k$ com raiz $R$, denotado por $S(T_k, R)$, é definido da seguinte forma recursiva:

  \begin{enumerate}
    \item Se $T_k$ é apenas o $k$-clique $R$, seu esqueleto é uma árvore com um único vértice $R$.
    \item Dada uma \emph{$k$-tree} enraizada $T_k$ com raiz $R$, obtida por $T_k'$ enraizada em $R$ através da adição de um novo vértice $v$ conectado a um $k$-clique $K$ (ver definição \ref{def:ktree}), seu esqueleto $S(T_k, R)$ é obtido adicionando a $S(T_k', R)$ um novo vértice $X = \{v\} \cup K$ e uma nova aresta $(X, Y)$, onde $Y$ é o vértice de $S(T_k', R)$ que contém $K$ com uma distância mínima da raiz. Chamamos $Y$ de pai de $X$.
  \end{enumerate}
\end{definition}

\begin{definition}[árvore característica]
  \label{def:chartree}
  \cite{caminiti} A árvore característica $T(T_k, R)$ de uma \emph{$k$-tree} enraizada $T_k$ com raiz $R$ é obtida rotulando os vértices e arestas de $S(T_k, R)$ da seguinte forma:

  \begin{enumerate}
    \item O vértice $R$ é rotulado $0$ e cada vértice $\{v\} \cup K$ é rotulado $v$;
    \item Cada aresta do vértice $\{v\} \cup K$ ao seu pai $\{v'\} \cup K'$ é rotulada com o índice do vértice em $K'$ (visualizando-o como um conjunto ordenado) que não aparece em $K$. Quando o pai é $R$ a aresta é rotulada $\varepsilon$.
  \end{enumerate}

  Note que a existência de um único vértice em $K' \setminus K$ é garantida pela definição \ref{def:skeleton}. De fato, $v'$ precisa aparecer em $K$, caso contrário $K' = K$ e o pai de $\{v'\} \cup K'$ contém $K$. Isso contradiz o fato de que cada vértice em $S(T_k, R)$ é ligado à distância mínima da raiz.
\end{definition}

A figura \ref{fig:transformation} mostra uma \emph{$k$-tree} de Rényi com $11$ vértices, seu esqueleto e sua árvore característica. O \emph{Dandelion Code} generalizado correspondente a essa árvore é $[(0, \varepsilon), (2, 0), (8, 2), (8, 1), (1, 2), (5, 2)]$.

\begin{figure}
  \begin{minipage}{0.3\textwidth}
    \centering
    \scalebox{0.75}{
      \begin{tikzpicture}
          [scale=.5,auto=left,every node/.style={draw, circle, inner sep = 0pt, minimum width = 0.7cm}]
        \node (n10) at (1,9) {3};
        \node[fill=gray!30] (n2) at (1.5,12) {9};
        \node (n1) at (2,3) {1};
        \node (n5) at (3,6) {5};
        \node (n7) at (3.5,0) {7};
        \node[fill=gray!30] (n9) at (4.5,12) {11};
        \node (n6) at (6,6) {6};
        \node (n4) at (9,3) {4};
        \node[fill=gray!30] (n3) at (7.5,12) {10};
        \node (n8) at (4.5,9) {8};
        \node (n11) at (8,9) {2};

        \foreach \from/\to in {n1/n5, n1/n7, n2/n5, n2/n8, n2/n9, n2/n10, n2/n11, n3/n8, n3/n9, n3/n10, n3/n11, n4/n9, n4/n11, n5/n7, n5/n8, n6/n8, n6/n9, n8/n9, n9/n10, n9/n11}
          \draw (\from) edge (\to);

        \draw (n1) edge [bend right=20] (n8);
        \draw (n1) edge [bend left=50] (n2);
        \draw (n2) edge [bend left] (n3);
        \draw (n2) edge [bend right=20] (n6);
        \draw (n3) edge [bend left] (n4);
        \draw (n3) edge [bend left=20] (n5);
        \draw (n7) edge [bend right=20] (n8);
      \end{tikzpicture}
    }

    (a)
  \end{minipage}\begin{minipage}{0.5\textwidth}
    \centering
    \scalebox{0.64}{\setstretch{1.0}
      \begin{tikzpicture}
          [scale=.6,auto=left,every node/.style={draw, rectangle, rounded corners = .1cm, inner sep = 0pt, minimum height = 1.25cm, minimum width = 2.25cm, align = center}]
        \node[fill=gray!30] (n0) at (5,12) {$\{9, 10, 11\}$};
        \node (n3) at (0.5,9) {$\{3\} \cup$ \\ $\{9, 10, 11\}$};
        \node (n8) at (5,9) {$\{8\} \cup$ \\ $\{9, 10, 11\}$};
        \node (n2) at (9.5,9) {$\{2\} \cup$ \\ $\{9, 10, 11\}$};
        \node (n5) at (2,6) {$\{5\} \cup$ \\ $\{8, 9, 10\}$};
        \node (n6) at (6.5,6) {$\{6\} \cup$ \\ $\{8, 9, 11\}$};
        \node (n4) at (11,6) {$\{4\} \cup$ \\ $\{2, 10, 11\}$};
        \node (n1) at (2,3) {$\{1\} \cup$ \\ $\{5, 8, 9\}$};
        \node (n7) at (2,0) {$\{5\} \cup$ \\ $\{1, 5, 8\}$};

      \foreach \from/\to in {n0/n3, n0/n8, n0/n2, n8/n5, n8/n6, n2/n4, n5/n1, n1/n7}
        \draw (\from) edge (\to);
      \end{tikzpicture}
    }

    (b)
  \end{minipage}\begin{minipage}{0.2\textwidth}
    \centering
    \begin{tikzpicture}
        [scale=.6,auto=left,every node/.style={draw, circle, inner sep = 0pt, minimum width = 0.65cm}]
      \node[fill=gray!30] (n0) at (5,8) {0};
      \node (n3) at (3.5,6) {3};
      \node (n8) at (5,6) {8};
      \node (n2) at (6.5,6) {2};
      \node (n5) at (3.75,4) {5};
      \node (n6) at (5.25,4) {6};
      \node (n4) at (6.75,4) {4};
      \node (n1) at (4,2) {1};
      \node (n7) at (4,0) {7};

      \foreach \from/\to/\elab in {n0/n3/$\varepsilon$, n0/n8/$\varepsilon$, n0/n2/$\varepsilon$, n8/n5/3, n8/n6/2, n2/n4/1, n5/n1/3, n1/n7/3}
        \draw (\from) -- (\to) node [draw=none, minimum width = 0.3cm, midway] {\scriptsize \elab};
    \end{tikzpicture}

    (c)
  \end{minipage}

  \caption{
    \textbf{(a)} Uma \emph{$3$-tree} de Rényi $R_3$ com 11 vértices e raiz $\{9, 10, 11\}$.
    \textbf{(b)} O esqueleto de $R_3$.
    \textbf{(c)} A árvore característica de $R_3$.
  }
  \label{fig:transformation}
\end{figure}

\subsection{Codificação}

O algoritmo para codificar uma \emph{$k$-tree} rotulada consiste em cinco passos. Aqui apresentamos esse algoritmo detalhando nossa implementação.

\begin{algorithm}[Algoritmo de codificação]
  \textbf{Entrada:} uma \emph{$k$-tree} $T_k$ com $n$ vértices\\
  \textbf{Saída:} um código em $\mathcal{A}^n_k$

  \begin{enumerate}
    \item Identificar $Q$, o $k$-clique adjacente à folha de maior rótulo $l_M$ de $T_k$;
    \item Através de um processo de re-rotulação $\phi$ (computado a partir de $Q$ e detalhado a seguir), transformar $T_k$ numa \emph{$k$-tree} de Rényi $R_k$;
    \item Gerar a árvore característica $T$ para $R_k$;
    \item Computar o \emph{Dandelion Code} generalizado $S$ para $T$;
    \item Remover da \emph{string} obtida $S$ o par correspondente a $\phi(l_M)$;
  \end{enumerate}

  O algoritmo retorna o par $(Q, S)$ computado durante esse processo.

  \vspace{2em}

  Na nossa implementação, uma \emph{$k$-tree} (estrutura definida no pacote {\tt ktree}) é representada através de uma lista de adjacências ({\tt Adj}) e um inteiro $k$ ({\tt K}). % TODO: é preciso definir lista de adjacência (em Fundamentos?).

  O algoritmo de codificação é implementado pela função {\tt CodingAlgorithm} do pacote {\tt codec}. A seguir, detalhamos os cinco passos.

  \begin{step}
    Primeiramente precisamos encontrar $l_M$, a folha de $T_k$ com maior rótulo. Uma folha em uma \emph{$k$-tree} consiste em um vértice de grau $k$, portanto basta iterar na lista de adjacências em ordem decrescente nos rótulos até encontrar um vértice com grau $k$. Isso foi implementado na função {\tt FindLm}, localizada no pacote {\tt ktree}.

    Encontrado $l_M$, atribuímos a $Q$ a lista {\tt Adj[$l_M$]} (ver função {\tt GetQ} do pacote {\tt ktree}.
  \end{step}

  \begin{step}
    Queremos transformar $T_k$ numa \emph{$k$-tree} de Rényi enraizada em $Q$. Para isso, precisamos definir uma permutação que associe os vértices de $Q$ a $\{n-k+1, n-k+2, \cdots, n\}$. A função de permutação, que chamamos de $\phi$, é definida da seguinte forma:

    \begin{enumerate}
      \item Se $q_i$ é o $i$-ésimo menor vértice em $Q$, fazemos $\phi(q_i) = n-k+i$;
      \item Para cada $q \not \in Q \cup \{n-k+1, \cdots, n\}$, fazemos $\phi(q) = q$;
      \item O restante dos valores são usados para fechar os ciclos de permutação, ou seja, para cada $q \in \{n-k+1, \cdots, n\} \setminus Q$, fazemos $\phi(q) = i$ tal que $\phi^j(i) = q$ e $j$ é maximizado.
    \end{enumerate}

    Essa computação é implementada pela função {\tt ComputePhi} no pacote {\tt ktree}.

    Usamos a função $\phi$ para re-rotular os vértices de $T_k$, obtendo a \emph{$k$-tree} de Rényi $R_k$. A implementação desse processo foi realizada na função {\tt Relabel} do pacote {\tt ktree}.

    A figura \ref{fig:phi} mostra uma representação gráfica da função $\phi$ usada para re-rotular a \emph{$3$-tree} mostrada na figura \ref{fig:rootedktree} com $Q = \{2, 3, 9\}$ produzindo a \emph{$k$-tree} de Rényi mostrada na figura \ref{fig:transformation}(a).

    \begin{figure}
      \centering
      \begin{tikzpicture}
          [scale=.6,auto=left,every node/.style={circle, inner sep = 0pt, minimum width = 0.55cm}]
        \node (n2) at (1,0) {2};
        \node (n9) at (2.5,0) {9};
        \node (n11) at (4,0) {11};

        \node (n3) at (6,0) {3};
        \node (n10) at (7.5,0) {10};

        \node (n1) at (9.5,0) {1};

        \node (n4) at (11.5,0) {4};

        \node (n5) at (13.5,0) {5};

        \node (n6) at (15.5,0) {6};

        \node (n7) at (17.5,0) {7};

        \node (n8) at (19.5,0) {8};

        \foreach \from/\to in {n2/n9, n9/n11, n11/n2, n3/n10, n10/n3}
          \draw (\from) edge[->,bend right] (\to);

        \foreach \vv in {n1, n4, n5, n6, n7, n8}
          \draw (\vv) edge[loop] (\vv);
      \end{tikzpicture}
      % TODO: melhorar desenho

      \caption{Representação gráfica da função $\phi$ computada para a \emph{$3$-tree} mostrada na figura \ref{fig:rootedktree}.}
      \label{fig:phi}
    \end{figure}
  \end{step}

  \begin{step}
    As definições \ref{def:skeleton} e \ref{def:chartree} sugerem algoritmos triviais para gerar a árvore característica $T$ para a \emph{$k$-tree} de Rényi $R_k$ obtida no passo anterior por meio do seu esqueleto (o processo visto na figura \ref{fig:transformation}).

    Para garantir tempo linear, no entanto, o artigo de Caminiti et. al \cite{caminiti} sugere evitar a construção explícita do esqueleto $S(R_k)$ e construir os conjuntos de vértices e arestas de $T$ separadamente.

    Para computar o conjunto de vértices, identifica-se cliques maximais em $R_k$ através da poda sucessiva das $k$-folhas de $R_k$. Esse processo pode ser visto na função {\tt pruneRk} do pacote {\tt characteristic}. Para cada vértice $v$ podado, essa função guarda uma lista $K_v \subseteq Adj(v)$ dos exatamente $k$ vértices adjacentes a $v$ que ainda não foram podados.

    Ao fim desse processo, que tem complexidade $O(nk)$, a \emph{$k$-tree} de Rényi é reduzida apenas à sua raiz $R = \{n-k+1, \cdots, n\}$.

    A partir das listas $K_i$ ($i \in V$) e da ordem em que os vértices foram podados, constrói-se o conjunto das arestas num processo de complexidade $O(nk)$ detalhado no programa 7 do artigo \cite{caminiti} cuja implementação encontra-se na função {\tt addEdges} do pacote {\tt characteristic}.
  \end{step}

  \begin{step}
    A escrever (computar o \emph{Dandelion Code} generalizado $S$ para $T$) % TODO
  \end{step}

  \begin{step}
    A escrever (remover da \emph{string} obtida $S$ o par correspondente a $\phi(l_M)$) % TODO
  \end{step}

  A concluir. % TODO
\end{algorithm}

\subsection{Decodificação}

A escrever. % TODO

\section{Experimentos e resultados}

A escrever. % TODO


  \cleardoublepage

  \chapter{Aprendizado de redes bayesianas}
\label{cap:aprendizado}

Neste capítulo apresentamos o problema de aprender redes bayesianas e descrevemos um método desenvolvido por Nie \emph{et al.} \cite{maua} para resolvê-lo que é baseado na amostragem de \emph{$k$-trees}. Mostramos como a geração uniforme de \emph{$k$-trees} desenvolvida no capítulo \ref{cap:geracao} foi utilizada nesse processo e comparamos os resultados obtidos com os resultados de outros trabalhos.

\section{Motivação}

Aprender uma rede bayesiana se refere ao processo de inferir a estrutura (ou seja, o DAG) dela a partir de dados. Como mostra Chickering \cite{chickering}, este é um problema NP-completo.

O aprendizado de redes bayesianas costuma servir para realizar inferências em situações com incerteza. O artigo de Nie \emph{et al.} \cite{maua} mostra que tais inferências são NP-difíceis até mesmo aproximadamente e todos os algoritmos conhecidos (exatos e comprovadamente bons) têm uma complexidade de pior caso exponencial no \emph{treewidth}.

Além disso, resultados empíricos sugerem que limitar o \emph{treewidth} pode melhorar a performance dos modelos e há evidências de que limitar a \emph{treewidth} da estrutura de uma rede bayesiana não causa perdas significativas na expressividade do modelo para conjuntos de dados reais (também visto em \cite{maua}).

Por isso, estamos interessados em fixar $k$ e aprender redes bayesianas cuja estrutura tem \emph{treewidth} limitada a $k$.

\vspace{2em}

A fim de identificar o ``melhor'' DAG para um determinado conjunto de dados, vamos supôr que há uma função de \emph{score} $s(G)$ que atribui uma pontuação para cada DAG $G$ em tempo constante. Segundo \cite{nie}, as funções de \emph{score} costumam poder ser escritas como a soma de funções de \emph{score} locais, ou seja,

$$s(G) = \sum_{i \in N} s_i(X_{\pi_i}).$$

Para cada variável, sua pontuação só depende do seu conjunto de pais. Ou seja, nosso problema é encontrar $G^*$ tal que

$$G^* = \argmax_{G \in \mathcal{G}_{n,k}} \sum_{i \in N} s_i(\pi_i),$$

onde $\mathcal{G}_{n,k}$ é o conjunto de todos os DAGs de \emph{treewidth} não maiores que $k$.

Mesmo esse problema é NP-difícil, como mostram Korhonen e Parviainen \cite{korhonen}. Entretanto, o artigo \cite{maua} mostra um método aproximado para aprender redes bayesianas com \emph{treewidth} limitado que é baseado em amostrar \emph{$k$-trees} e encontrar DAGs cujo grafo moral é um subgrafo dessas \emph{$k$-trees}.

Tal método funciona com domínios grandes e \emph{treewidth} alto. No artigo é mostrado empiricamente que ele tem um desempenho muito bom numa coleção de conjuntos de dados públicos.

\section{Aprendizado por amostragem de \emph{$k$-trees}}

A continuar. % TODO

\section{Experimentos e resultados}

A continuar. % TODO

  \cleardoublepage

  \chapter{Conclusão}
\label{cap:conclusao}

Ainda não foi escrita. % TODO
% Falar sobre outros samplings (DPS)
% Discutir vantagens e desvantagens da linguagem durante implementação

  \cleardoublepage

\backmatter

  \bibliographystyle{plain}
  \bibliography{referencias}

\end{document}
